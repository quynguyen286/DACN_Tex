\subsection{Kiến trúc hệ thống và Công nghệ nền tảng}
\subsubsection{Kiến trúc tổng thể}
\indent Hệ thống Xanh Market được thiết kế theo mô hình kiến trúc phân lớp (Layered Architecture) kết hợp với hướng dịch vụ (Service-oriented Architecture), đảm bảo tính mô-đun hóa cao và khả năng mở rộng độc lập từng thành phần. Hệ thống được tổ chức thành 5 tầng logic chính:

\begin{itemize}
    \item \textbf{Lớp thiết bị (Device Layer):} Là tầng thấp nhất bao gồm các thiết bị IoT triển khai tại hiện trường nông trại:
    \begin{itemize}
        \item \textit{Trạm quan trắc (IoT Gateway):} Sử dụng ESP32 làm bộ xử lý trung tâm, tích hợp các cảm biến môi trường (DHT11, DS18B20, Soil Moisture, Light Sensor) và module truyền thông (WiFi, RS485).
        \item \textit{Thiết bị chấp hành (Actuators):} Bao gồm hệ thống tưới tiêu tự động với van điện từ, máy bơm và các cơ cấu điều khiển khác.
        \item \textit{Giao thức truyền thông:} Sử dụng MQTT qua WiFi để truyền dữ liệu thời gian thực và HTTP cho tải lên hình ảnh/video.
    \end{itemize}

    \item \textbf{Lớp tích hợp (Integration Layer):} Đóng vai trò là cầu nối giữa thiết bị và hệ thống backend:
    \begin{itemize}
        \item \textit{API Gateway:} Điểm vào duy nhất cho tất cả các yêu cầu từ phía client, thực hiện xác thực, định tuyến và cân bằng tải.
        \item \textit{Message Broker (RabbitMQ):} Hệ thống hàng đợi thông điệp để xử lý dữ liệu IoT bất đồng bộ, đảm bảo tính reliable và decoupling giữa producer và consumer.
        \item \textit{Data Ingestion Service:} Tiếp nhận và tiền xử lý dữ liệu từ MQTT broker trước khi lưu trữ.
    \end{itemize}

    \item \textbf{Lớp nghiệp vụ (Business Layer):} Chứa toàn bộ logic xử lý cốt lõi của hệ thống:
    \begin{itemize}
        \item \textit{User Management Service:} Quản lý tài khoản người dùng, phân quyền và xác thực.
        \item \textit{Farm Management Service:} Xử lý logic quản lý nông trại, thiết bị và cấu hình giám sát.
        \item \textit{IoT Data Processor:} Service chuyên biệt để xử lý dữ liệu cảm biến, áp dụng thuật toán phát hiện lỗi (RFE) và tạo cảnh báo.
        \item \textit{Analytics Engine:} Phân tích dữ liệu lịch sử để đưa ra insights và khuyến nghị cho người dùng.
    \end{itemize}

    \item \textbf{Lớp dữ liệu (Data Layer):} Quản lý lưu trữ và truy xuất dữ liệu với chiến lược đa mô hình:
    \begin{itemize}
        \item \textit{PostgreSQL:} Cơ sở dữ liệu quan hệ lưu trữ thông tin cấu trúc (users, farms, devices, configurations).
        \item \textit{TimescaleDB:} Cơ sở dữ liệu chuỗi thời gian chuyên dụng cho dữ liệu cảm biến với hiệu năng truy vấn cao.
        \item \textit{AWS S3:} Lưu trữ file tĩnh như hình ảnh, log files và báo cáo.
    \end{itemize}

    \item \textbf{Lớp ứng dụng (Application Layer):} Giao diện người dùng cuối:
    \begin{itemize}
        \item \textit{Web Admin Portal:} Giao diện quản trị chính với dashboard thời gian thực, biểu đồ trực quan hóa dữ liệu và công cụ quản lý hệ thống.
        \item \textit{RESTful APIs:} Cung cấp các endpoint chuẩn cho tích hợp với hệ thống bên thứ ba.
        \item \textit{Real-time Notifications:} Hệ thống cảnh báo tức thì qua WebSocket cho các sự kiện quan trọng.
    \end{itemize}
\end{itemize}

Kiến trúc này đảm bảo tính linh hoạt cao với khả năng mở rộng ngang (horizontal scaling) thông qua containerization (Docker) và orchestration tự động. Mô hình phân tán giúp hệ thống chịu tải tốt với hàng trăm thiết bị IoT đồng thời, đồng thời đảm bảo tính bảo mật với cơ chế xác thực đa lớp và mã hóa dữ liệu.

\subsubsection{Nền tảng ứng dụng}
\indent Hệ thống sử dụng giao diện Web Admin làm nền tảng chính cho việc quản trị và giám sát. Thiết kế web-based mang lại các ưu điểm sau:
\begin{itemize}
    \item \textbf{Truy cập đa nền tảng:} Web Admin có thể được truy cập từ bất kỳ thiết bị nào có trình duyệt web hiện đại (desktop, laptop, tablet), không yêu cầu cài đặt phần mềm đặc biệt.
    \item \textbf{Định danh và xác thực:} Sử dụng cơ chế đăng nhập dựa trên tài khoản quản trị viên với phân quyền chi tiết (Role-Based Access Control), đảm bảo tính bảo mật cao.
    \item \textbf{Thông báo thời gian thực:} Hệ thống tích hợp WebSocket để gửi cảnh báo và cập nhật dữ liệu thời gian thực trực tiếp trên giao diện web.
    \item \textbf{Công nghệ phát triển:} Giao diện được xây dựng bằng ReactJS với Vite, kết hợp với các thư viện trực quan hóa dữ liệu như Chart.js để tạo dashboard tương tác.
\end{itemize}

\subsubsection{Hạ tầng Backend và Quy trình DevOps}
\indent Hệ thống Backend được xây dựng trên nền tảng điện toán đám mây riêng (Private Cloud) tại phòng thí nghiệm HPC Lab để đảm bảo tính bảo mật và chủ quyền dữ liệu. Quy trình triển khai áp dụng các tiêu chuẩn công nghiệp:
\begin{itemize}
    \item \textbf{Ảo hóa và Container hóa:} Sử dụng Docker để đóng gói các dịch vụ, đảm bảo môi trường vận hành đồng nhất giữa phát triển và sản xuất.
    \item \textbf{CI/CD Pipeline:} Tích hợp quy trình Tích hợp liên tục và Triển khai liên tục (CI/CD) để tự động hóa việc cập nhật phần mềm.
    \item \textbf{Kiểm soát chất lượng mã nguồn:} Sử dụng SonarQube để tự động quét và phân tích mã nguồn, phát hiện các lỗ hổng bảo mật và đảm bảo chất lượng code trước khi triển khai.
    \item \textbf{Giám sát hệ thống (Monitoring):} Sử dụng bộ công cụ Grafana và Prometheus để theo dõi hiệu năng server, tài nguyên hệ thống (CPU, RAM, Disk) và trạng thái các dịch vụ theo thời gian thực.
\end{itemize}

\subsubsection{Phần cứng IoT và Giao thức truyền thông}
\indent Hệ thống phần cứng được thiết kế để hoạt động bền bỉ trong môi trường nông nghiệp:
\begin{itemize}
    \item \textbf{Datalogger:} được đặt trong một hộp điện được chế tạo riêng, được thiết kế để sử dụng ngoài trời trong nông nghiệp.
    \item \textbf{Cảm biến:} Hỗ trợ đa dạng các loại cảm biến từ cơ bản (DHT11, cảm biến mưa, ánh sáng) cho mô hình hộ gia đình đến các cảm biến chuyên dụng công nghiệp (pH, EC) cho trang trại lớn.
    \item \textbf{Giao thức:} Sử dụng giao chuẩn RS485 cho việc kết nối cảm biến nhằm giảm thiểu nhiễu và đảm bảo truyền tin ổn định trên khoảng cách xa trong trang trại.
\end{itemize}

\subsection{Bài toán phát hiện lỗi cảm biến trong hệ thống IoT}
\subsubsection{Tổng quan về lỗi cảm biến trong môi trường IoT}
\indent Trong bối cảnh Công nghiệp 4.0, các hệ thống Internet vạn vật (IoT) đóng vai trò trụ cột trong việc thu thập dữ liệu thời gian thực cho các ứng dụng từ sản xuất thông minh đến thiết bị bay không người lái (drones). Tuy nhiên, đặc thù của các thiết bị IoT là thường xuyên phải hoạt động trong các môi trường khắc nghiệt (nhiệt độ cao, độ ẩm lớn, nhiễu điện từ).
\indent Điều này dẫn đến vấn đề suy giảm chất lượng dữ liệu, biểu hiện qua các dạng lỗi cảm biến phổ biến như:
\begin{itemize}
    \item \textbf{Giá trị bất thường (Outliers/Spikes):} Các giá trị nhảy vọt đột ngột không phản ánh thực tế.
    \item \textbf{Trôi tín hiệu (Drift):} Giá trị đọc sai lệch dần theo thời gian so với giá trị thực.
    \item \textbf{Kẹt giá trị (Stuck-at):} Cảm biến trả về một giá trị không đổi trong thời gian dài.
\end{itemize}
\indent Việc phát hiện sớm các lỗi này là tối quan trọng để đảm bảo độ tin cậy của toàn hệ thống.

\subsubsection{Các phương pháp tiếp cận hiện có và hạn chế}
\indent Hiện nay, bài toán phát hiện lỗi cảm biến thường được giải quyết theo ba hướng chính, tuy nhiên mỗi hướng đều tồn tại những hạn chế khi áp dụng cho các thiết bị IoT có 
tài nguyên hạn chế (Low-power IoT devices):
\begin{itemize}
    \item \textbf{Học máy cổ điển (Classic ML):} Thường gặp khó khăn trong việc nắm bắt các mẫu lỗi phức tạp hoặc phi tuyến tính trong chuỗi thời gian.
    \item \textbf{Học sâu (Deep Learning - DL):} Các mô hình như Autoencoder hay LSTM dù mạnh mẽ nhưng lại là những "hộp đen" (black-box), thiếu tính giải thích. Quan trọng hơn, chúng đòi hỏi tài nguyên tính toán và năng lượng lớn, không phù hợp để triển khai tại biên (Edge/Node).
    \item \textbf{Phương pháp dựa trên tương quan:} Hoạt động kém hiệu quả trong môi trường động và thường bỏ qua các thông tin nội tại của từng dòng dữ liệu riêng lẻ.
\end{itemize}
\indent Từ những phân tích trên, đặt ra yêu cầu về một phương pháp trích xuất đặc trưng (Feature Extraction) chuyên biệt, vừa đảm bảo độ chính xác cao, vừa tối ưu hóa tài nguyên tính toán.

\subsubsection{Cơ sở lý thuyết của phương pháp Robust Feature Extractor (RFE)}
\indentĐể giải quyết bài toán trên, nghiên cứu đề xuất phương pháp luận Robust Feature Extractor (RFE). Thay vì tăng độ phức tạp của mô hình phân loại, RFE tập trung vào việc biến đổi dữ liệu thô thành các đặc trưng giàu thông tin dựa trên các nguyên lý thống kê và xử lý tín hiệu sau:
\begin{itemize}
    \item \textbf{Phân tích động lực thời gian (Temporal Dynamics):} Dữ liệu cảm biến là dữ liệu chuỗi thời gian, do đó trạng thái hiện tại có quan hệ mật thiết với các trạng thái trước đó. RFE khai thác yếu tố này qua:
    \begin{itemize}
        \item \textbf{Tốc độ thay đổi (Speed of Change):} Tính toán đạo hàm bậc nhất rời rạc của tín hiệu ($value_t - value_{t-1}$) nhằm nắm bắt vận tốc và hướng biến thiên, giúp phát hiện các điểm gai bất thường.
        \item \textbf{Bộ nhớ tạm thời (Temporal Memory):} Sử dụng các giá trị trễ (lagged values) để cung cấp ngữ cảnh ngắn hạn cho mô hình.
    \end{itemize}
    \item \textbf{Thống kê cục bộ đa quy mô (Multi-Scale Local Statistics):} Một điểm dữ liệu cần được đánh giá trong ngữ cảnh của các điểm lân cận. RFE áp dụng kỹ thuật Cửa sổ trượt (Sliding Window) với nhiều kích thước khác nhau (ví dụ: $w=5, 10, 20$) để tính toán các tham số thống kê (trung bình, độ lệch chuẩn, độ xiên...).
    \begin{itemize}
        \item Ý nghĩa: Giúp mô hình nhận diện hành vi của tín hiệu ở cả quy mô ngắn hạn (nhiễu tức thời) và trung hạn (xu hướng trôi).
    \end{itemize}
    \item \textbf{Phân tích biến động (Volatility Analysis):} Một đóng góp mới của RFE là việc áp dụng thống kê trượt trên chính đặc trưng "Tốc độ thay đổi".
    \begin{itemize}
        \item Khi cảm biến gặp lỗi hoặc nhiễu loạn, độ lệch chuẩn của tốc độ thay đổi sẽ tăng cao. Đặc trưng này giúp lượng hóa sự "bất ổn" (instability) của tín hiệu một cách trực quan.
    \end{itemize}
    \item \textbf{Làm trơn và khử nhiễu (Smoothing):} Sử dụng Trung bình động lũy thừa (EWMA - Exponentially Weighted Moving Average) để làm nổi bật xu hướng (trend) của dữ liệu và giảm thiểu tác động của nhiễu ngẫu nhiên, giúp phân biệt rõ hơn giữa nhiễu môi trường và lỗi thực sự của thiết bị.
\end{itemize}
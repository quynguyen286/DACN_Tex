% \subsection{Công nghệ phần cứng/vi điều khiển}
% \subsubsection{Vi điều khiển ESP32}
% ESP32 là dòng vi điều khiển (SoC - System on Chip) hiệu năng cao, giá thành thấp được phát triển bởi Espressif Systems, tích hợp sẵn kết nối Wi-Fi 2.4 GHz và Bluetooth chế độ 
% kép (Dual-mode Bluetooth). Được xây dựng dựa trên kiến trúc vi xử lý Xtensa® Dual-Core 32-bit LX6, ESP32 sở hữu khả năng xử lý vượt trội với xung nhịp có thể đạt tới 240 MHz và 
% hiệu suất tính toán lên đến 600 DMIPS. So với thế hệ tiền nhiệm ESP8266, ESP32 không chỉ vượt trội về tốc độ xử lý mà còn được nâng cấp mạnh mẽ về số lượng chân giao tiếp (GPIO), 
% dung lượng bộ nhớ (520 KB SRAM) và các ngoại vi tích hợp như cảm biến chạm (Touch Sensor), cảm biến Hall và bộ điều khiển CAN. Với sự hỗ trợ mạnh mẽ của hệ điều hành thời gian 
% thực FreeRTOS ngay từ tầng phần cứng, ESP32 trở thành nền tảng lý tưởng cho các ứng dụng IoT phức tạp đòi hỏi khả năng đa nhiệm (Multitasking) và xử lý dữ liệu thời gian thực.\cite{ESP32_docs}

% \subsection{Công nghệ cảm biến/ngoại vi}
% \subsubsection{Giao thức I2C (Inter-Integrated Circuit)}
% I2C (Inter-Integrated Circuit) là một giao thức truyền thông nối tiếp đồng bộ, bán song công (Half - duplex) được phát triển bởi Philips Semiconductor (nay là NXP), đóng vai trò 
% là chuẩn giao tiếp tầm ngắn phổ biến nhất trong các thiết kế vi mạch tích hợp. Đặc trưng của I2C là kiến trúc Multi-Master/Multi-Slave, cho phép kết nối nhiều thiết bị ngoại vi 
% (Cảm biến, EEPROM, RTC) với vi điều khiển chỉ thông qua hai đường dây tín hiệu duy nhất: SDA (Serial Data) để truyền tải dữ liệu hai chiều và SCL (Serial Clock) để đồng bộ hóa 
% xung nhịp. Nhờ cơ chế định địa chỉ mềm (Addressing) thay vì sử dụng các chân chọn chip (Chip Select) vật lý như giao thức SPI, I2C giúp tối thiểu hóa số lượng chân GPIO cần 
% thiết và giảm độ phức tạp của mạch in, trở thành giải pháp lý tưởng cho các ứng dụng đo lường và giám sát trong các hệ thống IoT.\cite{I2C_spec}

% \subsubsection{Giao thức DVPI (Digital Video Port Interface)}
% DVPI (Digital Video Port Interface) là một giao diện truyền thông song song (Parallel Interface) tiêu chuẩn được sử dụng rộng rãi để kết nối các cảm biến hình ảnh 
% CMOS (như OV2640) với vi điều khiển hoặc bộ xử lý trung tâm. Khác với các giao thức truyền nối tiếp như SPI hay I2C vốn bị giới hạn về băng thông, DVP sử dụng một bus dữ liệu 
% rộng (thường là 8bit hoặc 10bit) để truyền tải toàn bộ một byte dữ liệu điểm ảnh (pixel) trong mỗi chu kỳ xung nhịp. Cơ chế vận hành của DVP dựa vào sự đồng bộ hóa chính xác 
% giữa ba tín hiệu điều khiển cốt lõi: PCLK (Pixel Clock) xác định tốc độ lấy mẫu từng điểm ảnh, VSYNC (Vertical Sync) báo hiệu thời điểm bắt đầu một khung hình mới, và HREF/HSYNC 
% (Horizontal Reference) xác định dữ liệu hợp lệ của từng dòng quét ngang. Nhờ kiến trúc này, DVP cung cấp băng thông đủ lớn để truyền tải dữ liệu hình ảnh thời gian thực với độ 
% trễ thấp, làm nền tảng cho các ứng dụng thị giác máy tính trên thiết bị nhúng.\cite{Camera_DVPI_spec}

% \subsubsection{GPIO (General Purpose Input/Output)}
% GPIO (General Purpose Input/Output) là các chân tín hiệu số đa năng trên vi điều khiển, đóng vai trò là giao diện vật lý tương tác trực tiếp với các mạch điện tử bên ngoài. 
% Không bị giới hạn bởi một chức năng cố định, trạng thái của chân GPIO có thể được lập trình viên cấu hình linh hoạt ở hai chế độ: Output (Đầu ra) để điều khiển mức logic điện 
% áp (High/Low) nhằm kích hoạt các cơ cấu chấp hành, hoặc Input (Đầu vào) để đọc trạng thái logic (0/1) từ các công tắc hay cảm biến số. Trên nền tảng ESP32, kiến trúc GPIO Matrix 
% cho phép ánh xạ linh hoạt các tín hiệu ngoại vi tới các chân vật lý khác nhau, giúp tối ưu hóa quá trình thiết kế mạch in (PCB) và định tuyến tín hiệu.\cite{GPIO_spec}

% \subsubsection{ADC (Analog-to-Digital Converter)} 
% ADC (Analog-to-Digital Converter) là thành phần ngoại vi thiết yếu giúp vi điều khiển (vốn chỉ hoạt động với các tín hiệu số rời rạc) có thể 'hiểu' được các đại lượng vật lý 
% biến thiên liên tục trong môi trường thực như nhiệt độ, ánh sáng hay độ ẩm. Nguyên lý hoạt động của ADC dựa trên quá trình Lấy mẫu (Sampling) và Lượng tử hóa (Quantization): 
% nó đo điện áp đầu vào tại các khoảng thời gian rời rạc và chuyển đổi giá trị điện áp đó thành một chuỗi số nhị phân tương ứng với độ phân giải của bộ chuyển đổi.\cite{GPIO_spec}

% \subsection{Công nghệ hạ tầng mạng}
% \subsubsection{Giao thức WiFi (IEEE 802.11)}
% WiFi (Wireless Fidelity) là một họ các giao thức mạng không dây cục bộ (WLAN) dựa trên bộ tiêu chuẩn IEEE 802.11, cho phép các thiết bị điện tử trao đổi dữ liệu qua sóng vô 
% tuyến tốc độ cao. Trong các ứng dụng IoT, WiFi đóng vai trò là lớp truyền dẫn (Transport Layer) mạnh mẽ, cung cấp khả năng kết nối trực tiếp vào hạ tầng Internet thông qua các 
% thiết bị định tuyến (Router/Access Point) mà không cần Gateway chuyển đổi giao thức phức tạp. ESP32 hỗ trợ chuẩn 802.11 b/g/n hoạt động trên băng tần 2.4 GHz, cung cấp sự cân 
% bằng tối ưu giữa phạm vi phủ sóng (xuyên vật cản tốt hơn 5GHz) và tốc độ truyền tải dữ liệu, đáp ứng tốt các yêu cầu từ gửi gói tin điều khiển nhỏ đến truyền tải hình ảnh dung 
% lượng lớn.\cite{ESP32_docs}

% \subsubsection{Giao thức MQTT}
% MQTT (Message Queuing Telemetry Transport) là một giao thức truyền thông điệp theo mô hình Xuất bản - Đăng ký (Publish - Subscribe) cực kỳ nhẹ, hoạt động trên nền tảng TCP/IP. 
% Được thiết kế chuyên biệt cho các môi trường có băng thông thấp, độ trễ cao hoặc mạng không ổn định, MQTT đã trở thành tiêu chuẩn vàng trong các hệ thống Internet vạn vật (IoT). 
% Khác với mô hình Request - Response truyền thống của HTTP, kiến trúc MQTT tách biệt hoàn toàn giữa thiết bị gửi (Publisher) và thiết bị nhận (Subscriber) thông qua một thành phần 
% trung gian gọi là Broker. Broker đóng vai trò như một 'bưu điện số', chịu trách nhiệm lọc, định tuyến và phân phối các gói tin dựa trên các chủ đề (Topics) phân cấp. Nhờ cơ chế 
% đóng gói tiêu đề (header) tối giản chỉ 2 byte và khả năng duy trì kết nối liên tục (Keep - alive), MQTT giúp tối ưu hóa năng lượng tiêu thụ cho các thiết bị nhúng như ESP32, đồng 
% thời đảm bảo độ tin cậy cao trong việc truyền tải dữ liệu cảm biến và lệnh điều khiển thời gian thực.\cite{mqtt_spec}

% \subsubsection{Giao thức HTTP}
% HTTP (Hypertext Transfer Protocol) là giao thức truyền tải siêu văn bản hoạt động ở Lớp Ứng dụng (Application Layer) của mô hình OSI, đóng vai trò là nền tảng cốt lõi của việc 
% trao đổi dữ liệu trên World Wide Web. HTTP vận hành dựa trên mô hình Client - Server theo cơ chế Request - Response (Yêu cầu - Phản hồi): Client (trong trường hợp này là ESP32) gửi 
% một yêu cầu HTTP đến Server, và Server sẽ trả về một mã trạng thái kèm theo dữ liệu tương ứng. Đặc điểm nổi bật của HTTP là tính 'phi trạng thái' (Stateless), nghĩa là mỗi cặp 
% yêu cầu - phản hồi là độc lập và Server không lưu giữ thông tin về Client giữa các lần kết nối (trừ khi sử dụng các cơ chế bổ sung như Cookies hay Token). Trong các hệ thống IoT 
% hiện đại, HTTP thường được triển khai dưới dạng kiến trúc RESTful API, cho phép các thiết bị nhúng tương tác với cơ sở dữ liệu và dịch vụ đám mây thông qua các phương thức chuẩn 
% hóa như GET, POST, PUT và DELETE.\cite{http_rfc}

% \subsection{Công nghệ phần mềm/Firmware}
% \subsubsection{Hệ điều hành FreeRTOS}
% FreeRTOS (Real-Time Operating System) là một nhân hệ điều hành thời gian thực mã nguồn mở, được thiết kế chuyên biệt cho các vi điều khiển nhúng để quản lý tài nguyên phần cứng 
% và lập lịch tác vụ. Khác với các hệ điều hành đa dụng (như Windows hay Linux) tập trung vào trải nghiệm người dùng, mục tiêu tối thượng của FreeRTOS là tính xác định 
% (Determinism) và độ trễ thấp, đảm bảo các tác vụ quan trọng phải được thực thi trong một khoảng thời gian quy định nghiêm ngặt. Trên nền tảng ESP32, FreeRTOS được tích hợp sâu 
% vào bộ công cụ phát triển (ESP-IDF/Arduino Core), cung cấp cơ chế lập lịch ưu tiên (Preemptive Scheduling) cho phép phân tách một chương trình lớn thành các tiểu trình độc lập 
% gọi là Tác vụ (Tasks). Mỗi tác vụ sở hữu ngăn xếp (Stack) và mức độ ưu tiên riêng biệt, giúp hệ thống vận hành đa nhiệm (Multitasking) mượt mà ngay cả trên các thiết bị giới hạn 
% về tài nguyên.\cite{rtos}

% \subsubsection{Định dạng dữ liệu JSON}
% JSON (JavaScript Object Notation) là một chuẩn định dạng trao đổi dữ liệu văn bản mở (Open Standard), gọn nhẹ và độc lập với ngôn ngữ lập trình, được sử dụng rộng rãi để lưu 
% trữ và truyền tải dữ liệu có cấu trúc. Mặc dù có nguồn gốc từ cú pháp đối tượng của JavaScript, JSON hiện nay được hỗ trợ bởi hầu hết các ngôn ngữ lập trình hiện đại (C/C++, 
% Python, Java, Dart...). Cấu trúc của JSON được xây dựng dựa trên hai cấu trúc dữ liệu phổ quát: tập hợp các cặp 'tên : giá trị' (Key-Value pairs) tạo thành một Đối tượng 
% (Object), và danh sách các giá trị có thứ tự tạo thành một Mảng (Array). Trong các hệ thống IoT, JSON đóng vai trò là 'ngôn ngữ chung' (Lingua Franca), cho phép Firmware nhúng 
% (viết bằng C++) đóng gói dữ liệu cảm biến thành chuỗi văn bản tiêu chuẩn trước khi truyền qua giao thức MQTT hoặc HTTP tới máy chủ.\cite{json_spec}

% \subsubsection{Giao thức UART}
% UART (Universal Asynchronous Receiver-Transmitter) là giao thức truyền thông nối tiếp không đồng bộ, đóng vai trò là chuẩn giao tiếp cơ bản nhất giữa máy tính và các hệ thống 
% nhúng. Khác với I2C hay SPI vốn hoạt động dựa trên tín hiệu xung nhịp (Clock) đồng bộ, UART truyền tải dữ liệu theo cơ chế bất đồng bộ (Asynchronous). Điều này nghĩa là bên 
% gửi (Transmitter - TX) và bên nhận (Receiver - RX) không chia sẻ chung một đường dây Clock, mà thay vào đó, chúng phải thống nhất trước với nhau về tốc độ truyền tải, gọi là 
% Baud Rate (tốc độ Baud). Dữ liệu được đóng gói thành các khung (Frame) bao gồm bit bắt đầu (Start bit), các bit dữ liệu (Data bits), bit kiểm tra chẵn lẻ tùy chọn (Parity bit) 
% và bit kết thúc (Stop bit). Trên ESP32, UART không chỉ là giao diện lập trình nạp firmware mà còn là kênh giao tiếp quan trọng để xuất nhật ký hoạt động (System Logs) phục vụ 
% quá trình giám sát và gỡ lỗi.\cite{GPIO_spec}

% \subsection{Thuật toán RFE (Robust Feature Extractor)}

% Trong kỷ nguyên Công nghiệp 4.0, Internet vạn vật (IoT) đóng vai trò trụ cột với các ứng dụng trải rộng từ sản xuất thông minh đến các hệ thống bay không người lái 
% nhằm thu thập dữ liệu thời gian thực. Tuy nhiên, do thường xuyên phải vận hành trong các môi trường khắc nghiệt chịu ảnh hưởng bởi nhiệt độ, độ ẩm hay nhiễu điện từ, 
% dữ liệu cảm biến dễ gặp phải các sai lệch nghiêm trọng như giá trị bất thường, trôi tín hiệu hoặc bị kẹt giá trị. Các giải pháp hiện hữu vẫn tồn tại nhiều rào cản đáng kể: 
% trong khi học máy cổ điển gặp khó khăn với các mẫu lỗi phức tạp và phương pháp dựa trên tương quan tỏ ra kém hiệu quả trong môi trường động, thì học sâu (Deep Learning) – 
% dù mạnh mẽ – lại đòi hỏi chi phí huấn luyện lớn, thiếu tính minh bạch (hộp đen) và dễ bỏ sót các lỗi hiếm gặp. Bối cảnh này đặt ra yêu cầu cấp thiết về một thuật toán phát 
% hiện lỗi mới có khả năng lấp đầy 'khoảng trống' hiện tại: vừa đảm bảo hiệu suất cao và khả năng diễn giải, vừa tối ưu hóa tài nguyên tính toán để phù hợp với đặc thù năng 
% lượng thấp của các thiết bị IoT.\cite{RFE_spec}

% Thay vì tập trung phát triển các mô hình phân loại phức tạp, nghiên cứu đề xuất phương pháp luận Robust Feature Extractor (RFE), một cách tiếp cận ưu tiên việc xử lý dữ liệu 
% đầu vào thông minh nhằm biến đổi chuỗi thời gian thô thành không gian đặc trưng đa chiều giàu thông tin. RFE tích hợp các nhóm đặc trưng chiến lược bao gồm: tín hiệu gốc và 
% tốc độ thay đổi để nắm bắt động lực học tức thời; bộ nhớ tạm thời (temporal memory) thông qua các giá trị trễ để cung cấp ngữ cảnh ngắn hạn; và xu hướng được làm mịn bằng 
% phương pháp EWMA. Điểm đột phá của RFE nằm ở việc áp dụng thống kê trượt trên tốc độ thay đổi nhằm định lượng mức độ biến động (volatility) của tín hiệu, kết hợp với thống kê 
% cục bộ đa quy mô để phân tích hành vi dữ liệu trên nhiều cửa sổ trượt khác nhau. Đặc biệt, thiết kế của RFE chủ động loại bỏ các đặc trưng dựa trên tương quan, một chiến lược 
% then chốt giúp giảm thiểu chi phí tính toán và ngăn chặn hiện tượng quá khớp (overfitting), đảm bảo tính hiệu quả khi triển khai trên các thiết bị IoT.

% Quá trình thực nghiệm được tiến hành trên bộ dữ liệu SeDa thu thập từ cảm biến DHT11 nhằm đánh giá toàn diện hiệu năng của phương pháp Robust Feature Extractor (RFE) thông qua 
% việc đối sánh với ba kỹ thuật cơ sở là Autoencoder (AE), Stationary Wavelet Transform (SWT) và TsAssure trên 7 mô hình học máy tiêu chuẩn (như Random Forest, SVM, XGBoost...). 
% Kết quả định lượng cho thấy bộ đặc trưng RFE mang lại sự vượt trội tuyệt đối về mọi chỉ số hiệu năng, duy trì độ chính xác ổn định ở mức cao từ 90,7\% đến 92,9\%, áp đảo hoàn toàn 
% so với sự thiếu ổn định của các phương pháp đối chứng như TsAssure (chỉ đạt 85 - 88\%) hay Autoencoder (79 - 82\%). Không chỉ tối ưu về độ chính xác, RFE còn chứng minh ưu thế vượt 
% bậc về hiệu quả tính toán với thời gian trích xuất đặc trưng chỉ 2,05 giây — nhanh gấp 5 lần so với Autoencoder (10,97 giây) và tiệm cận tốc độ của TsAssure—qua đó khẳng định 
% tính khả thi cao khi triển khai cho các ứng dụng phát hiện lỗi thời gian thực trên các thiết bị IoT giới hạn tài nguyên.

% Nghiên cứu nhấn mạnh vai trò tiên quyết của kỹ thuật trích xuất đặc trưng (feature engineering), khẳng định rằng việc khai thác sâu động lực thời gian và biến động cục bộ mang 
% lại hiệu suất vượt trội so với các mô hình 'hộp đen' phức tạp như Autoencoder. Phương pháp RFE được xác định là giải pháp tối ưu đạt được sự cân bằng lý tưởng giữa độ chính xác 
% cao và chi phí tính toán thấp, chứng minh tính khả thi vượt trội khi triển khai trên các hệ thống IoT giới hạn tài nguyên năng lượng. Trên cơ sở đó, các hướng phát triển tương 
% lai được đề xuất tập trung vào việc tinh giản số lượng đặc trưng nhằm giảm thiểu độ trễ dự đoán, đồng thời mở rộng phạm vi kiểm thử phương pháp trên đa dạng các loại cảm biến 
% và kịch bản lỗi thực tế khác nhau.

\subsection{Tổng quan về nông nghiệp thông minh (Smart Farm)}
\subsubsection{Khái niệm và bối cảnh}
\indent Nông nghiệp thông minh (Smart Farming) là việc ứng dụng các công nghệ hiện đại của Công nghiệp 4.0 vào quy trình sản xuất nông nghiệp. Trọng tâm của mô hình này là khả 
năng thu thập, xử lý và phân tích dữ liệu thời gian thực từ các cảm biến nông nghiệp nhằm đưa ra các quyết định canh tác chính xác. Mục tiêu là tạo ra một khung làm việc 
(framework) bền vững và có khả năng mở rộng, giúp tích hợp các thiết bị IoT với nền tảng xử lý dữ liệu để hỗ trợ tưới tiêu thích ứng và cập nhật thông tin sản phẩm.
\indent Trong bối cảnh hiện nay, nông nghiệp thông minh không chỉ dừng lại ở tự động hóa sản xuất mà còn mở rộng sang việc kết nối thông minh giữa người nông dân và khách hàng, 
tạo ra chuỗi giá trị minh bạch từ trang trại đến bàn ăn.

\subsubsection{Các thành phần công nghệ cốt lõi}
\indent Dựa trên thực tế triển khai các giải pháp nông nghiệp hiện đại, một hệ thống Smart Farm tiêu chuẩn bao gồm các thành phần chính sau:
\begin{itemize}
    \item \textbf{Hệ thống giám sát môi trường (Environmental Monitoring):} Sử dụng mạng lưới các cảm biến chuyên dụng để đo đạc các thông số vi khí hậu và môi trường đất/nước. Các thiết bị điển hình bao gồm cảm biến độ ẩm đất, cảm biến nhiệt độ - độ ẩm (như DHT11), cảm biến ánh sáng, cảm biến mưa, và các cảm biến đo chất lượng dinh dưỡng như pH, EC/TDS.
    \item \textbf{Hạ tầng thu thập và xử lý dữ liệu (Data Infrastructure):} Dữ liệu từ cảm biến được thu thập liên tục và chuyển về các Nền tảng luồng dữ liệu (Data Streaming Platform - DSP). Tại đây, dữ liệu được xử lý để cung cấp thông tin chi tiết (actionable insights) cho người dùng cuối.
    \item \textbf{Nền tảng tương tác người dùng:} Các ứng dụng di động (ví dụ: Mini App trên nền tảng Zalo) đóng vai trò là giao diện điều khiển, cho phép nông dân truy cập dữ liệu trang trại thời gian thực, nhận cảnh báo và thực hiện các tác vụ như kích hoạt tưới tiêu từ xa. Đồng thời, công nghệ cũng hỗ trợ nông dân trong việc tiếp thị kỹ thuật số và quản lý nhật ký canh tác.
\end{itemize}

\subsubsection{Thách thức trong triển khai thực tế}
\indent Mặc dù mang lại nhiều lợi ích, việc triển khai hệ thống IoT trong nông nghiệp đối mặt với những thách thức lớn về độ bền và độ tin cậy:
\begin{itemize}
    \item \textbf{Điều kiện vận hành khắc nghiệt:} Các thiết bị IoT và bộ ghi dữ liệu (Datalogger) phải hoạt động liên tục ngoài trời, chịu tác động trực tiếp của thời tiết khắc nghiệt, đòi hỏi các tiêu chuẩn cao về khả năng chống nước, cách điện và bảo vệ cơ học.
    \item \textbf{Yêu cầu về độ ổn định dữ liệu:} Để đảm bảo hoạt động sản xuất an toàn và tin cậy, hệ thống cần duy trì khả năng thu thập dữ liệu ổn định và giảm thiểu nhiễu tín hiệu trong quá trình truyền thông.
\end{itemize}

\subsection{Kiến trúc hệ thống và Công nghệ nền tảng}
\subsubsection{Kiến trúc tổng thể}
\indent Hệ thống Xanh Market được thiết kế theo mô hình phân tán, tích hợp chặt chẽ giữa các thiết bị IoT tại hiện trường, nền tảng xử lý dữ liệu dòng (Data Streaming Platform - DSP) và các ứng dụng người dùng cuối. Kiến trúc tổng thể bao gồm các thành phần chính:
\begin{itemize}
    \item \textbf{Lớp thiết bị (Device Layer):} Bao gồm các trạm quan trắc (IoT Box) và hệ thống điều khiển tưới tiêu được triển khai thực tế tại nông trại (như Tomochan Farm). Các thiết bị này thu thập dữ liệu môi trường (nhiệt độ, độ ẩm, ánh sáng, EC/TDS) và gửi về hệ thống trung tâm.
    \item \textbf{Lớp xử lý và lưu trữ (Processing and Storage Layer):} Sử dụng Nền tảng luồng dữ liệu (DSP) để thu thập, xử lý và phân tích dữ liệu thời gian thực từ cảm biến. Hệ thống Backend được triển khai theo kiến trúc hiện đại, đảm bảo khả năng mở rộng và chịu lỗi.
    \item \textbf{Lớp ứng dụng (Application Layer):} Bao gồm cổng thông tin quản trị (CMS Portal) dành cho quản trị viên và các ứng dụng Zalo Mini App dành cho nông dân và khách hàng.
\end{itemize}

\subsubsection{Nền tảng ứng dụng}
\indent Thay vì phát triển ứng dụng di động truyền thống (Native App), dự án lựa chọn nền tảng Zalo Mini App làm giao diện chính cho người dùng cuối với các ưu điểm vượt trội về khả năng tiếp cận thị trường Việt Nam:
\begin{itemize}
    \item \textbf{Tận dụng lượng người dùng sẵn có:} Zalo là nền tảng chat phổ biến nhất Việt Nam, giúp ứng dụng tiếp cận ngay lập tức với nông dân và khách hàng mà không cần cài đặt phức tạp.
    \item \textbf{Định danh và xác thực:} Sử dụng cơ chế đăng nhập qua Zalo (Zalo Login), giúp hệ thống truy xuất được thông tin người dùng thực đã được xác minh, giảm thiểu rủi ro tài khoản ảo.
    \item \textbf{Thông báo tích hợp (Zalo Notification):} Hệ thống gửi cảnh báo và tin tức nông nghiệp trực tiếp qua tin nhắn Zalo, đảm bảo thông tin đến người dùng nhanh chóng và thuận tiện.
    \item \textbf{Công nghệ phát triển:} Zalo Mini App được xây dựng dựa trên các công nghệ web phổ biến (React + Vite), giúp tối ưu hóa quy trình phát triển và bảo trì.
\end{itemize}

\subsubsection{Hạ tầng Backend và Quy trình DevOps}
\indent Hệ thống Backend được xây dựng trên nền tảng điện toán đám mây riêng (Private Cloud) tại phòng thí nghiệm HPC Lab để đảm bảo tính bảo mật và chủ quyền dữ liệu. Quy trình triển khai áp dụng các tiêu chuẩn công nghiệp:
\begin{itemize}
    \item \textbf{Ảo hóa và Container hóa:} Sử dụng Docker để đóng gói các dịch vụ, đảm bảo môi trường vận hành đồng nhất giữa phát triển và sản xuất.
    \item \textbf{CI/CD Pipeline:} Tích hợp quy trình Tích hợp liên tục và Triển khai liên tục (CI/CD) để tự động hóa việc cập nhật phần mềm.
    \item \textbf{Kiểm soát chất lượng mã nguồn:} Sử dụng SonarQube để tự động quét và phân tích mã nguồn, phát hiện các lỗ hổng bảo mật và đảm bảo chất lượng code trước khi triển khai.
    \item \textbf{Giám sát hệ thống (Monitoring):} Sử dụng bộ công cụ Grafana và Prometheus để theo dõi hiệu năng server, tài nguyên hệ thống (CPU, RAM, Disk) và trạng thái các dịch vụ theo thời gian thực.
\end{itemize}

\subsubsection{Tích hợp trí tuệ nhân tạo}
\indent Một điểm nhấn công nghệ của hệ thống là việc tích hợp mô hình ngôn ngữ lớn (LLM) để hỗ trợ nông dân tạo nội dung số:
\begin{itemize}
    \item \textbf{Mô hình:} Sử dụng mô hình mã nguồn mở Llama 3.2 (3 tỷ tham số) từ Meta.
    \item \textbf{Triển khai:} Mô hình được chạy cục bộ (On-premise) thông qua Ollama trên máy chủ có GPU, đảm bảo dữ liệu của nông dân không bị gửi ra các dịch vụ bên thứ ba, tăng cường tính bảo mật.
    \item \textbf{Tự động hóa:} Quy trình tạo bài viết (Blog generation) được quản lý bởi công cụ n8n, kết nối giữa ứng dụng Zalo và mô hình AI để tự động soạn thảo các bài quảng bá nông trại dựa trên từ khóa đầu vào.
\end{itemize}

\subsubsection{Phần cứng IoT và Giao thức truyền thông}
\indent Hệ thống phần cứng được thiết kế để hoạt động bền bỉ trong môi trường nông nghiệp:
\begin{itemize}
    \item \textbf{Datalogger:} được đặt trong một hộp điện được chế tạo riêng, được thiết kế để sử dụng ngoài trời trong nông nghiệp.
    \item \textbf{Cảm biến:} Hỗ trợ đa dạng các loại cảm biến từ cơ bản (DHT11, cảm biến mưa, ánh sáng) cho mô hình hộ gia đình đến các cảm biến chuyên dụng công nghiệp (pH, EC) cho trang trại lớn.
    \item \textbf{Giao thức:} Sử dụng giao chuẩn RS485 cho việc kết nối cảm biến nhằm giảm thiểu nhiễu và đảm bảo truyền tin ổn định trên khoảng cách xa trong trang trại.
\end{itemize}

\subsection{Bài toán phát hiện lỗi cảm biến trong hệ thống IoT}
\subsubsection{Tổng quan về lỗi cảm biến trong môi trường IoT}
\indent Trong bối cảnh Công nghiệp 4.0, các hệ thống Internet vạn vật (IoT) đóng vai trò trụ cột trong việc thu thập dữ liệu thời gian thực cho các ứng dụng từ sản xuất thông minh đến thiết bị bay không người lái (drones). Tuy nhiên, đặc thù của các thiết bị IoT là thường xuyên phải hoạt động trong các môi trường khắc nghiệt (nhiệt độ cao, độ ẩm lớn, nhiễu điện từ).
\indent Điều này dẫn đến vấn đề suy giảm chất lượng dữ liệu, biểu hiện qua các dạng lỗi cảm biến phổ biến như:
\begin{itemize}
    \item \textbf{Giá trị bất thường (Outliers/Spikes):} Các giá trị nhảy vọt đột ngột không phản ánh thực tế.
    \item \textbf{Trôi tín hiệu (Drift):} Giá trị đọc sai lệch dần theo thời gian so với giá trị thực.
    \item \textbf{Kẹt giá trị (Stuck-at):} Cảm biến trả về một giá trị không đổi trong thời gian dài.
\end{itemize}
\indent Việc phát hiện sớm các lỗi này là tối quan trọng để đảm bảo độ tin cậy của toàn hệ thống.

\subsubsection{Các phương pháp tiếp cận hiện có và hạn chế}
\indent Hiện nay, bài toán phát hiện lỗi cảm biến thường được giải quyết theo ba hướng chính, tuy nhiên mỗi hướng đều tồn tại những hạn chế khi áp dụng cho các thiết bị IoT có 
tài nguyên hạn chế (Low-power IoT devices):
\begin{itemize}
    \item \textbf{Học máy cổ điển (Classic ML):} Thường gặp khó khăn trong việc nắm bắt các mẫu lỗi phức tạp hoặc phi tuyến tính trong chuỗi thời gian.
    \item \textbf{Học sâu (Deep Learning - DL):} Các mô hình như Autoencoder hay LSTM dù mạnh mẽ nhưng lại là những "hộp đen" (black-box), thiếu tính giải thích. Quan trọng hơn, chúng đòi hỏi tài nguyên tính toán và năng lượng lớn, không phù hợp để triển khai tại biên (Edge/Node).
    \item \textbf{Phương pháp dựa trên tương quan:} Hoạt động kém hiệu quả trong môi trường động và thường bỏ qua các thông tin nội tại của từng dòng dữ liệu riêng lẻ.
\end{itemize}
\indent Từ những phân tích trên, đặt ra yêu cầu về một phương pháp trích xuất đặc trưng (Feature Extraction) chuyên biệt, vừa đảm bảo độ chính xác cao, vừa tối ưu hóa tài nguyên tính toán.

\subsubsection{Cơ sở lý thuyết của phương pháp Robust Feature Extractor (RFE)}
\indentĐể giải quyết bài toán trên, nghiên cứu đề xuất phương pháp luận Robust Feature Extractor (RFE). Thay vì tăng độ phức tạp của mô hình phân loại, RFE tập trung vào việc biến đổi dữ liệu thô thành các đặc trưng giàu thông tin dựa trên các nguyên lý thống kê và xử lý tín hiệu sau:
\begin{itemize}
    \item \textbf{Phân tích động lực thời gian (Temporal Dynamics):} Dữ liệu cảm biến là dữ liệu chuỗi thời gian, do đó trạng thái hiện tại có quan hệ mật thiết với các trạng thái trước đó. RFE khai thác yếu tố này qua:
    \begin{itemize}
        \item \textbf{Tốc độ thay đổi (Speed of Change):} Tính toán đạo hàm bậc nhất rời rạc của tín hiệu ($value_t - value_{t-1}$) nhằm nắm bắt vận tốc và hướng biến thiên, giúp phát hiện các điểm gai bất thường.
        \item \textbf{Bộ nhớ tạm thời (Temporal Memory):} Sử dụng các giá trị trễ (lagged values) để cung cấp ngữ cảnh ngắn hạn cho mô hình.
    \end{itemize}
    \item \textbf{Thống kê cục bộ đa quy mô (Multi-Scale Local Statistics):} Một điểm dữ liệu cần được đánh giá trong ngữ cảnh của các điểm lân cận. RFE áp dụng kỹ thuật Cửa sổ trượt (Sliding Window) với nhiều kích thước khác nhau (ví dụ: $w=5, 10, 20$) để tính toán các tham số thống kê (trung bình, độ lệch chuẩn, độ xiên...).
    \begin{itemize}
        \item Ý nghĩa: Giúp mô hình nhận diện hành vi của tín hiệu ở cả quy mô ngắn hạn (nhiễu tức thời) và trung hạn (xu hướng trôi).
    \end{itemize}
    \item \textbf{Phân tích biến động (Volatility Analysis):} Một đóng góp mới của RFE là việc áp dụng thống kê trượt trên chính đặc trưng "Tốc độ thay đổi".
    \begin{itemize}
        \item Khi cảm biến gặp lỗi hoặc nhiễu loạn, độ lệch chuẩn của tốc độ thay đổi sẽ tăng cao. Đặc trưng này giúp lượng hóa sự "bất ổn" (instability) của tín hiệu một cách trực quan.
    \end{itemize}
    \item \textbf{Làm trơn và khử nhiễu (Smoothing):} Sử dụng Trung bình động lũy thừa (EWMA - Exponentially Weighted Moving Average) để làm nổi bật xu hướng (trend) của dữ liệu và giảm thiểu tác động của nhiễu ngẫu nhiên, giúp phân biệt rõ hơn giữa nhiễu môi trường và lỗi thực sự của thiết bị.
\end{itemize}
\subsubsection{Use case: Monitor System Analytics}
\begin{figure}[H]
    \centering
    \includegraphics[width=0.9\textwidth]{img/analytics.png}
    \caption{Sơ đồ luồng quản lý và tạo báo cáo}
\end{figure}
\begin{table}[H]
    \centering
    \small
    \renewcommand{\arraystretch}{1.3} % Tăng khoảng cách dòng cho dễ đọc
    \begin{tabular}{|p{3.2cm}|p{11.5cm}|}
        \hline
        \textbf{Mã số usecase} & UC-01: Monitor System Analytics \\
        \hline
        \textbf{Tên usecase} & Giám sát Phân tích Hệ thống \\
        \hline
        \textbf{Mô tả} & Admin theo dõi tổng quan các chỉ số hoạt động, xếp hạng hiệu suất nông trại và thống kê thiết bị trên Dashboard. \\
        \hline
        \textbf{Actor} & System Admin \\
        \hline
        \textbf{Tiền điều kiện} & Tài khoản Admin đã đăng nhập và có quyền truy cập module Analytics. \\
        \hline
        \textbf{Hậu điều kiện} & Dữ liệu thống kê được hiển thị đầy đủ và cập nhật mới nhất. \\
        \hline
        \textbf{Trigger} & Admin nhấn chọn menu ``Analytics \& Insights'' trên thanh điều hướng. \\
        \hline
        \textbf{Luồng chính} &
        \begin{enumerate}[leftmargin=*]
            \item Admin truy cập màn hình Analytics.
            \item Hệ thống kiểm tra quyền xem của Admin.
            \item Hệ thống tải dữ liệu tổng hợp từ cơ sở dữ liệu.
            \item Hệ thống hiển thị Dashboard Metrics (doanh thu, sản lượng, nhiệt độ trung bình).
            \item Hệ thống hiển thị Farm Performance Rankings (xếp hạng hiệu suất).
            \item Hệ thống hiển thị Device Statistics (biểu đồ trạng thái Online/Offline).
        \end{enumerate} \\
        \hline
        \textbf{Quy tắc nghiệp vụ} &
        \begin{itemize}[leftmargin=*]
            \item \textbf{BR-01 (Data Latency):} Dữ liệu hiển thị trên Dashboard phải được cập nhật gần thời gian thực (Real-time), độ trễ tối đa không quá 30 giây.
            \item \textbf{BR-02 (Default View):} Mặc định hiển thị dữ liệu tổng hợp của toàn hệ thống trong 7 ngày gần nhất.
            \item \textbf{BR-03 (Access Control):} Chỉ tài khoản Admin có quyền ``Manage'' mới hiển thị nút cấu hình ``Configure Alert Thresholds''.
            \item \textbf{BR-04 (Ranking Logic):} Xếp hạng nông trại dựa trên chỉ số KPI tổng hợp (tỷ lệ sản lượng / mức tiêu thụ năng lượng).
        \end{itemize} \\
        \hline
        \textbf{Luồng thay thế / Mở rộng} &
        \begin{itemize}[leftmargin=*]
            \item \textbf{E-01 (Filter):} Tại bước 4-6, Admin chọn bộ lọc thời gian hoặc khu vực \textrightarrow{} Hệ thống truy vấn lại và cập nhật hiển thị.
            \item \textbf{E-02 (Export):} Admin bấm ``Export'' \textrightarrow{} Hệ thống kiểm tra định dạng file (PDF/CSV) và tiến hành tải xuống.
            \item \textbf{E-03 (Alert Config):} Admin thay đổi ngưỡng cảnh báo \textrightarrow{} Hệ thống lưu quy tắc mới vào cơ sở dữ liệu và áp dụng ngay lập tức.
        \end{itemize} \\
        \hline
    \end{tabular}
\end{table}

\newpage

\subsubsection{Use case: Troubleshoot System}
\begin{figure}[H]
    \centering
    \begin{subfigure}{0.95\textwidth}
        \centering
        \includegraphics[width=0.95\textwidth]{img/troubleshoot.png}
        \caption{Sơ đồ xử lý sự cố và theo dõi tình trạng hệ thống}
    \end{subfigure}
\end{figure}
\noindent Hình mô tả lần lượt quá trình kiểm tra dịch vụ, khắc phục từ xa, và cách truy xuất log, xem lịch sử cảnh báo để đối soát.

\begin{table}[H]
    \centering
    \small
    \renewcommand{\arraystretch}{1.3}
    \begin{tabular}{|p{3.2cm}|p{11.5cm}|}
        \hline
        \textbf{Mã số usecase} & UC-02: Troubleshoot System \\
        \hline
        \textbf{Tên usecase} & Kiểm tra và xử lý sự cố \\
        \hline
        \textbf{Mô tả} & Quy trình xem tình trạng sức khỏe hệ thống, đọc log chi tiết và gửi lệnh khắc phục sự cố từ xa. \\
        \hline
        \textbf{Actor} & System Admin \\
        \hline
        \textbf{Tiền điều kiện} & Hệ thống phát hiện sự cố (Alert) hoặc Admin thực hiện kiểm tra định kỳ. \\
        \hline
        \textbf{Hậu điều kiện} & Nguyên nhân lỗi được xác định hoặc lệnh sửa chữa đã được gửi đi. \\
        \hline
        \textbf{Trigger} & Admin nhận được thông báo lỗi hoặc truy cập menu ``Troubleshoot''. \\
        \hline
        \textbf{Luồng chính} &
        \begin{enumerate}[leftmargin=*]
            \item Admin truy cập trang Troubleshoot.
            \item Hệ thống hiển thị đèn trạng thái (Health Indicators) của Server, API và Mạng.
            \item Hệ thống truy xuất và hiển thị danh sách 50 dòng System Logs gần nhất.
            \item Admin phân tích các chỉ số màu đỏ (Lỗi) và nội dung log để xác định nguyên nhân.
        \end{enumerate} \\
        \hline
        \textbf{Quy tắc nghiệp vụ} &
        \begin{itemize}[leftmargin=*]
            \item \textbf{BR-05 (Health Coding):} Trạng thái hệ thống phải được mã hóa màu: Xanh (Ổn định), Vàng (Cảnh báo tải >80\%), Đỏ (Lỗi/Ngừng hoạt động).
            \item \textbf{BR-06 (Log Privacy):} Các thông tin nhạy cảm trong log (như Password, API Key) phải được che dấu (masking) bằng ký tự `******' trước khi hiển thị.
            \item \textbf{BR-07 (Action Permission):} Chỉ tài khoản Super Admin mới có quyền thực thi các lệnh tác động hệ thống như ``Restart Service''.
        \end{itemize} \\
        \hline
        \textbf{Luồng thay thế / Mở rộng} &
        \begin{itemize}[leftmargin=*]
            \item \textbf{E-01 (Perform Diagnostics):} Admin chọn thiết bị lỗi \textrightarrow{} chọn hành động (Ping/Restart Service/Telemetry) \textrightarrow{} Hệ thống thực thi lệnh và trả về kết quả (Success/Timeout).
            \item \textbf{E-02 (Export Logs):} Admin bấm ``Export Logs'' \textrightarrow{} Hệ thống tổng hợp log lỗi thành file CSV/JSON và kích hoạt tải xuống.
            \item \textbf{E-03 (Filter Logs):} Admin lọc theo mức độ ``Error'' \textrightarrow{} Hệ thống ẩn các log thông thường, chỉ hiện log lỗi.
        \end{itemize} \\
        \hline
    \end{tabular}
\end{table}

\newpage

\subsubsection{Use case: Manage Reports}
\begin{figure}[H]
    \centering
    \includegraphics[width=0.9\textwidth]{img/reports.png}
    \caption{Sơ đồ luồng quản lý và tạo báo cáo}
\end{figure}
\noindent Hình thể hiện quy trình xem danh sách báo cáo, tạo mới, lập lịch hoặc tạo báo cáo tùy chỉnh và cập nhật lại danh sách.

\begin{table}[H]
    \centering
    \small
    \renewcommand{\arraystretch}{1.3}
    \begin{tabular}{|p{3.2cm}|p{11.5cm}|}
        \hline
        \textbf{Mã số usecase} & UC-03: Manage Reports \\
        \hline
        \textbf{Tên usecase} & Quản lý và tạo báo cáo \\
        \hline
        \textbf{Mô tả} & Admin xem danh sách báo cáo đã lưu, tạo báo cáo mới tức thì hoặc lập lịch gửi báo cáo tự động. \\
        \hline
        \textbf{Actor} & System Admin \\
        \hline
        \textbf{Tiền điều kiện} & Admin đã đăng nhập thành công. \\
        \hline
        \textbf{Hậu điều kiện} & Danh sách báo cáo được cập nhật; file báo cáo được tạo ra. \\
        \hline
        \textbf{Trigger} & Admin truy cập trang ``Reports''. \\
        \hline
        \textbf{Luồng chính} &
        \begin{enumerate}[leftmargin=*]
            \item Admin mở trang Reports.
            \item Hệ thống truy vấn cơ sở dữ liệu báo cáo.
            \item Hệ thống hiển thị danh sách Recent Reports (bao gồm: Tên, Ngày tạo, Người tạo, Loại báo cáo).
            \item Admin xem thông tin hoặc tải về các báo cáo cũ.
        \end{enumerate} \\
        \hline
        \textbf{Quy tắc nghiệp vụ} &
        \begin{itemize}[leftmargin=*]
            \item \textbf{BR-08 (Retention Policy):} Báo cáo chỉ được lưu trữ trực tuyến trong 90 ngày. Sau thời gian này, dữ liệu sẽ được chuyển sang lưu trữ lạnh (Archive).
            \item \textbf{BR-09 (File Format):} Hệ thống phải hỗ trợ xuất báo cáo ra hai định dạng: PDF (để in ấn/trình ký) và Excel/CSV (để phân tích số liệu).
            \item \textbf{BR-10 (Schedule Limit):} Mỗi tài khoản Admin chỉ được thiết lập tối đa 5 lịch báo cáo tự động để đảm bảo hiệu năng hệ thống.
        \end{itemize} \\
        \hline
        \textbf{Luồng thay thế / Mở rộng} &
        \begin{itemize}[leftmargin=*]
            \item \textbf{E-01 (Generate):} Admin bấm ``Generate New'' \textrightarrow{} Hệ thống thu thập dữ liệu hiện tại và tạo file báo cáo ngay lập tức.
            \item \textbf{E-02 (Schedule):} Admin bấm ``Schedule'' \textrightarrow{} Hệ thống hiển thị form chọn tần suất (hàng ngày/tuần) \textrightarrow{} Lưu lịch chạy tự động (Cron job).
            \item \textbf{E-03 (Custom Report):} Admin chọn các trường dữ liệu tùy chỉnh \textrightarrow{} Bấm ``Create'' \textrightarrow{} Hệ thống tạo báo cáo theo mẫu riêng.
        \end{itemize} \\
        \hline
    \end{tabular}
\end{table}
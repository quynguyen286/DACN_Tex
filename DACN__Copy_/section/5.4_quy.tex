\subsubsection{Quản lý lược đồ dữ liệu cảm biến từ xa}
\begin{itemize}
    \item \textbf{Mục tiêu:} đảm bảo rằng mọi dữ liệu (telemetry) gửi từ hàng ngàn cảm biến về máy chủ đều:
    \begin{itemize}
        \item \textbf{Thống nhất (Consistent):} Dữ liệu luôn tuân theo một định dạng chuẩn.
        \item \textbf{Chất lượng (Valid):} Dữ liệu không bị sai, thiếu hoặc lỗi định dạng.
        \item \textbf{Dễ dàng mở rộng (Extensible):} Dễ dàng thêm cảm biến mới hoặc phiên bản firmware mới mà không làm sập hệ thống.
        \item \textbf{Dễ bảo trì (Maintainable):} Khi schema thay đổi (ví dụ: thêm cảm biến đo độ pH), máy chủ biết cách xử lý phiên bản cũ và mới.
    \end{itemize}
    \item \textbf{Đề xuất cấu trúc:} Sử dụng định dạng JSON vì tính linh hoạt và dễ tiếp cận. Cấu trúc được chia thành 3 schema chính:
    \begin{itemize}
        \item \textbf{Schema chung \textit{(device\_base):}} Tất cả các gói tin \textit{(packet)} đều phải chứa các thông tin cơ bản này để định tuyến và nhận diện.
        \begin{itemize}
            \item \texttt{device\_id(String):} Mã định danh thiết bị.
            \item \texttt{timestamp(int64):} Thời gian gửi dữ liệu.
            \item \texttt{schema\_id(String):} Tên của schema mà gói tin này đang sử dụng.   
            \item \texttt{schema\_version(String):} Phiên bản của schema.
        \end{itemize}
        \item \textbf{Schema 1 (env\_data):} Dữ liệu môi trường.
            \begin{verbatim}
                {
                    "device_id": "env-sensor-zone-a-01",
                    "timestamp": 1678886400000,
                    "schema_id": "env_data",
                    "schema_version": "v1.1",
                    "data": {
                        "temperature_celsius": 28.5,
                        "humidity_percent": 75.2
                    }
                }
            \end{verbatim}
        \item \textbf{Schema 2 (camera\_data):} Dữ liệu hình ảnh từ camera.
            \begin{verbatim}
                {
                    "device_id": "cam-zone-a-01",
                    "timestamp": 1678887000000,
                    "schema_id": "camera_event",
                    "schema_version": "v1.0",
                    "data": {
                        "image_url": "https://storage.server.com/images/12345.jpg",
                        "file_size": 51200
                    }
                }
            \end{verbatim}
        \item \textbf{Schema 3 (device\_health):} Dữ liệu tình trạng thiết bị.
            \begin{verbatim}
                {
                    "device_id": "cam-zone-a-01",
                    "timestamp": 1678886500000,
                    "schema_id": "device_health",
                    "schema_version": "v1.0",
                    "data": {
                        "status": "online",       // "online", "offline", "error"
                        "battery_percent": 85.0,  // If battery-powered
                        "uptime_seconds": 3600,
                        "error_code": 0           // 0 = OK, optional error codes
                    }
                }
            \end{verbatim}
    \end{itemize}
    \item \textbf{Đề xuất hệ thống quản lý:} Cách máy chủ biết \texttt{env\_data v1.0} và \texttt{v1.1} khác gì nhau:
    \begin{itemize}
        \item \textbf{Kho lưu trữ Schema (Schema Registry):}
        \begin{itemize}
            \item Là một cơ sở dữ liệu (hoặc một Git repository) chứa các tệp tin định nghĩa schema.
            \item Sẽ có các tệp như là \texttt{"env\_data\_v1.0.0.json"}, \texttt{"camera\_event\_v1.0.0.json"},....
            \item Ưu điểm: Cả team phát triển thiết bị (firmware) và team phát triển phần mềm (software) đều nhìn vào đây để làm việc.
        \end{itemize}
        \item \textbf{Quản lý Phiên bản (Versioning):} Sử dụng Semantic Versioning (vMAJOR.MINOR.PATCH).
        \begin{itemize}
            \item PATCH (v1.0.1): Sửa lỗi nhỏ, không ảnh hưởng cấu trúc.
            \item MINOR (v1.1.0): Thêm trường dữ liệu mới, vẫn tương thích ngược.
            \item MAJOR (v2.0.0): Thay đổi lớn, không tương thích ngược.
        \end{itemize}
        \item \textbf{Xác thực Schema (Schema Validation)}
        \begin{itemize}
            \item Dữ liệu từ cảm biến gửi đến (ví dụ: qua MQTT Broker).
            \item Một dịch vụ "Ingestor" (bộ tiếp nhận) sẽ đọc gói tin.
            \item Nó thấy \texttt{schema\_id}: \texttt{"env\_data"} và \texttt{schema\_version: "v1.1"}.
            \item Nó lập tức tra cứu trong Schema Registry để lấy tệp định nghĩa \texttt{env\_data\_v1.1.json}.	
            \item Nó dùng tệp định nghĩa này để xác thực (validate) gói tin nhận được.
            \begin{itemize}
                \item Nếu hợp lệ: Đẩy dữ liệu vào database (ví dụ: TimeScaleDB, InfluxDB).
                \item Nếu không hợp lệ: Gói tin bị loại bỏ và gửi cảnh báo (ví dụ: "Thiết bị 'env-sensor-zone-a-01' đang gửi dữ liệu rác!").
            \end{itemize}
        \end{itemize}
    \end{itemize}
    \item \textbf{Ưu điểm của đề xuất này:}
    \begin{itemize}
        \item \textbf{Ngăn chặn dữ liệu rác:} Hệ thống tự động loại bỏ dữ liệu sai định dạng ngay từ đầu vào.
        \item \textbf{Gỡ lỗi dễ dàng:} Biết chính xác thiết bị nào đang gửi sai phiên bản schema.
        \item \textbf{Dễ dàng mở rộng:} Khi muốn thêm cảm biến độ ẩm đất (v1.1), các thiết bị v1.0 cũ vẫn hoạt động bình thường song song với các thiết bị v1.1 mới.
        \item \textbf{Tính độc lập:} Xử lý camera (dữ liệu nặng) riêng biệt với telemetry (dữ liệu nhẹ) giúp hệ thống nhanh và ổn định.
    \end{itemize}
\end{itemize}

\subsubsection{Đề xuất cải tiến cho thuật toán RFE}

\subsubsubsection{Vấn đề và hạn chế của RFE nguyên bản}
\indent Thuật toán Robust Feature Extractor (RFE) nguyên bản hoạt động dựa trên cơ chế phân tích chuỗi thời gian đơn biến (Univariate Time-series Analysis). Tức là, việc đánh giá trạng thái của một cảm biến chỉ dựa vào dữ liệu quá khứ của chính nó.
\indent Mặc dù hiệu quả trong việc phát hiện các lỗi cục bộ (như gai nhiễu, kẹt giá trị), phương pháp này tồn tại một hạn chế lớn khi gặp các biến động môi trường đồng loạt.
\begin{itemize}
    \item Ví dụ: Khi hệ thống nhà màng mở lưới che nắng vào buổi trưa, nhiệt độ của toàn bộ khu vườn sẽ tăng vọt trong thời gian ngắn.
    \item Vấn đề: RFE cơ bản sẽ ghi nhận "Tốc độ thay đổi" ($v_t$) tăng cao đột ngột và có thể nhầm lẫn đây là lỗi trôi (Drift) hoặc gai (Spike), dẫn đến cảnh báo sai (False Positive).
\end{itemize}

\subsubsubsection{Giải pháp đề xuất:}
\indent Để khắc phục hạn chế trên, nghiên cứu đề xuất mở rộng vector đặc trưng bằng cách bổ sung thông tin tương quan không gian. Ý tưởng cốt lõi là so sánh hành vi của cảm biến mục tiêu với các cảm biến lân cận trong cùng một khu vực canh tác.
\indent Vector đặc trưng đầu ra $F_t$ của thuật toán cải tiến sẽ có dạng:
$$F_t = [\underbrace{x_t, v_t, \sigma_t, EWMA_t, Lags}_{\text{Temporal}}, \underbrace{S_t}_{\text{Spatial}}]$$
Trong đó, $S_t$ là chỉ số Độ lệch Không gian (Spatial Deviation), được tính toán nhằm lượng hóa mức độ khác biệt của cảm biến hiện tại so với tập thể.

\subsubsubsection{Cơ chế hoạt động}
\indent Giả sử tại thời điểm $t$, ta có giá trị của cảm biến mục tiêu là $x_t$ và tập hợp giá trị của $k$ cảm biến lân cận là $N_t = \{y_t^{(1)}, y_t^{(2)}, ..., y_t^{(k)}\}$.
\indent Chỉ số $S_t$ được tính toán đơn giản theo công thức:
$$S_t = |x_t - \text{median}(N_t)|$$
Trong đó: 
$$\text{median}(N) = \begin{cases} y_{(\frac{k+1}{2})} & \text{nếu } k \text{ lẻ} \\ \frac{1}{2} (y_{(\frac{k}{2})} + y_{(\frac{k}{2} + 1)}) & \text{nếu } k \text{ chẵn} \end{cases}$$
(Sử dụng trung vị (median) thay vì trung bình (mean) để tăng cường khả năng chịu đựng giá trị ngoại lai (Outliers).).

\indent Về lí do sử dụng median thay vì mean:
\begin{itemize}
    \item Mean tính toán dựa trên độ lớn của tất cả các phần tử, nên nó rất nhạy cảm với nhiễu. Một giá trị nhiễu lớn có thể kéo lệch toàn bộ giá trị tham chiếu.
    \item Median là một đại lượng thống kê thứ tự, có tính kháng nhiễu cao hơn hẳn. Nó coi các giá trị lỗi đột biến là các phần tử nằm ở rìa của phân phối và loại bỏ chúng, giúp thuật toán không bị báo động giả.
\end{itemize}

\subsubsubsection{Kết luận}
\indent Việc chuyển đổi từ phân tích đơn biến sang đa biến thông qua chỉ số $S_t$ mang lại hai lợi ích thiết thực cho hệ thống Smart Farm:
\begin{itemize}
    \item \textbf{Giảm tỷ lệ báo động giả:} Phân biệt rõ ràng giữa sự cố thiết bị và các hiện tượng thời tiết cực đoan.
    \item \textbf{Tăng độ tin cậy:} Hệ thống tận dụng được lợi thế của mạng lưới thiết bị IoT dày đặc để tự kiểm chứng chéo dữ liệu lẫn nhau ngay tại biên trước khi gửi về máy chủ.
\end{itemize}

\indent Hạn chế của phương pháp cải tiến đề xuất: 
\begin{itemize}
    \item Mặc dù sử dụng bộ lọc trung vị (Median Filter) giúp tăng cường khả năng kháng nhiễu so với trung bình cộng, phương pháp này vẫn tồn tại giới hạn về điểm gãy. Theo lý thuyết thống kê, điểm gãy của trung vị là 50\%. Điều này có nghĩa là nếu hơn 50\% số lượng cảm biến tham chiếu trong mạng lưới gặp sự cố đồng thời và sai lệch theo cùng một hướng, giá trị trung vị sẽ bị sai lệch theo, dẫn đến khả năng tính toán chỉ số tương quan không gian $S_t$ không còn chính xác.
    \item Tuy nhiên, trong thực tế vận hành hệ thống IoT nông nghiệp:
    \begin{itemize}
        \item Các lỗi cảm biến phần cứng thường mang tính ngẫu nhiên và độc lập, xác suất xảy ra đồng thời trên diện rộng là thấp.
        \item Trường hợp các giá trị thay đổi đồng loạt thường là dấu hiệu của sự thay đổi môi trường thực tế hoặc sự cố cấp hệ thống (mất nguồn), lúc này hệ thống giám sát cần chuyển sang cơ chế cảnh báo mức độ cao hơn thay vì phát hiện lỗi mức node.
    \end{itemize}
\end{itemize}
\indent Hướng khắc phục: Để giải quyết triệt để vấn đề này, các nghiên cứu tiếp theo có thể tích hợp cơ chế trọng số tin cậy. Hệ thống sẽ đánh giá độ tin cậy của từng node lân cận dựa trên lịch sử hoạt động của chúng. Các node có tiền sử lỗi thường xuyên sẽ bị giảm trọng số hoặc loại bỏ khỏi tập tham chiếu khi tính toán Median. 

\subsubsubsection{Firmware}
\begin{table}[h!]
\centering
\caption{Bảng phân tích rủi ro và phương án giảm thiểu}
\label{tab:rui_ro}
\renewcommand{\arraystretch}{1.3} % Giãn dòng
\small % Giảm cỡ chữ xuống một chút để bảng gọn hơn

% Chỉnh lại độ rộng cột cho vừa khít khổ giấy A4
\begin{tabular}{|c|>{\raggedright\arraybackslash}p{4.5cm}|c|>{\raggedright\arraybackslash}p{7cm}|}
\hline
\centering\textbf{STT} & \centering\textbf{Rủi ro (Risk)} & \textbf{Mức độ} & \centering\textbf{Phương án giảm thiểu} \tabularnewline
\hline

1 & 
\textbf{Hỏng hóc phần cứng:} \newline Cháy ESP32 hoặc Camera do đấu sai nguồn/ngắn mạch. & 
\centering Cao & 
\begin{itemize}[nosep, leftmargin=*]
    \item Mua dự phòng 01 bộ linh kiện.
    \item Kiểm tra kỹ mạch bằng VOM trước khi cấp nguồn.
\end{itemize} \\
\hline

2 & 
\textbf{Tràn bộ nhớ (Stack Overflow):} \newline Do xử lý ảnh JPEG lớn trên FreeRTOS. & 
\centering Cao & 
\begin{itemize}[nosep, leftmargin=*]
    \item Sử dụng hàm: \newline \texttt{uxTaskGetStackHighWaterMark} để giám sát RAM.
    \item Cân nhắc sử dụng ESP32 có PSRAM.
\end{itemize} \\
\hline

3 & 
\textbf{Nhiễu tín hiệu:} \newline Camera bị sọc hoặc cảm biến đất báo giá trị ảo. & 
\centering Trung bình & 
\begin{itemize}[nosep, leftmargin=*]
    \item Sử dụng tụ lọc nguồn 100uF và 100nF.
    \item Đi dây tín hiệu ngắn gọn, tách biệt với dây động cơ.
\end{itemize} \\
\hline

4 & 
\textbf{Độ trễ truyền ảnh cao:} \newline Video bị giật, lag khi mạng WiFi yếu. & 
\centering Trung bình & 
\begin{itemize}[nosep, leftmargin=*]
    \item Giảm độ phân giải ảnh (VGA/QVGA) khi mạng kém.
    \item Tách luồng gửi ảnh sang Core 0 độc lập.
\end{itemize} \\
\hline

5 & 
\textbf{Chậm tiến độ:} \newline Do vướng mắc thuật toán phức tạp. & 
\centering Thấp & 
\begin{itemize}[nosep, leftmargin=*]
    \item Ưu tiên xử lý phần Camera trước (phần khó nhất).
    \item Tham khảo cộng đồng Open Source ESP32.
\end{itemize} \\
\hline

\end{tabular}
\end{table}






\subsubsubsection{Software}
\begin{table}[h!]
\centering
\caption{Phân tích rủi ro phần mềm và phương án xử lý}
\label{tab:rui_ro_sw}
\renewcommand{\arraystretch}{1.3} 

\begin{tabular}{|c|p{4.5cm}|c|p{6.5cm}|}
\hline
\textbf{STT} & \centering\textbf{Rủi ro (Risk)} & \textbf{Mức độ} & \centering\textbf{Phương án giảm thiểu} \tabularnewline
\hline

1 & 
\textbf{Nghẽn cổ chai Database:} \newline Khi hàng trăm thiết bị gửi dữ liệu cùng lúc, việc Insert từng dòng gây chậm. & 
Cao & 
- Sử dụng kỹ thuật Batch Insert (gom 100-500 bản ghi/lần). \newline
- Tận dụng kiến trúc Hypertables của TimescaleDB. \\
\hline

2 & 
\textbf{Mất dữ liệu tại Broker:} \newline RabbitMQ bị đầy hàng đợi (Queue overflow) nếu Backend xử lý chậm. & 
Trung bình & 
- Cấu hình cơ chế Message Acknowledgment (chỉ xóa khi đã xử lý). \newline
- Tăng số lượng Worker (Consumer) trong NestJS. \\
\hline

3 & 
\textbf{Dữ liệu rác (Invalid Schema):} \newline Thiết bị gửi sai định dạng JSON làm lỗi parser. & 
Cao & 
- Triển khai lớp Schema Validation (DTO) chặt chẽ tại đầu vào. \newline
- Ghi log gói tin lỗi ra bảng riêng để debug. \\
\hline

4 & 
\textbf{Truy vấn Dashboard chậm:} \newline Load biểu đồ lịch sử 1 năm mất quá nhiều thời gian ($>$5s). & 
Trung bình & 
- Sử dụng Materialized Views để tính sẵn giá trị trung bình theo giờ. \newline
- Cache kết quả truy vấn vào Redis. \\
\hline

5 & 
\textbf{Lỗ hổng bảo mật API:} \newline Người dùng truy cập vào dữ liệu nông trại của người khác. & 
Cao & 
- Áp dụng chặt chẽ Guards và Interceptors. \newline
- Kiểm tra quyền sở hữu (Ownership check) trong mọi API. \\
\hline

\end{tabular}
\end{table}

\subsubsubsection{Firmware}
\begin{table}[h!]
\centering
\caption{Bảng phân tích rủi ro và phương án giảm thiểu}
\label{tab:rui_ro}
\renewcommand{\arraystretch}{1.3} % Giãn dòng
\small % Giảm cỡ chữ xuống một chút để bảng gọn hơn

% Chỉnh lại độ rộng cột cho vừa khít khổ giấy A4
\begin{tabular}{|c|>{\raggedright\arraybackslash}p{4.5cm}|c|>{\raggedright\arraybackslash}p{7cm}|}
\hline
\centering\textbf{STT} & \centering\textbf{Rủi ro (Risk)} & \textbf{Mức độ} & \centering\textbf{Phương án giảm thiểu} \tabularnewline
\hline

1 & 
\textbf{Hỏng hóc phần cứng:} \newline Cháy ESP32 hoặc Camera do đấu sai nguồn/ngắn mạch. & 
\centering Cao & 
\begin{itemize}[nosep, leftmargin=*]
    \item Mua dự phòng 01 bộ linh kiện.
    \item Kiểm tra kỹ mạch bằng VOM trước khi cấp nguồn.
\end{itemize} \\
\hline

2 & 
\textbf{Độ trễ truyền ảnh cao:} \newline Video bị giật, lag khi mạng WiFi yếu. & 
\centering Trung bình & 
\begin{itemize}[nosep, leftmargin=*]
    \item Tách luồng gửi ảnh sang Core 0 độc lập.
\end{itemize} \\
\hline

3 & 
\textbf{Chậm tiến độ:} \newline Do vướng mắc thuật toán phức tạp. & 
\centering Thấp & 
\begin{itemize}[nosep, leftmargin=*]
    \item Ưu tiên xử lý phần Camera trước (phần khó nhất).
    \item Tham khảo cộng đồng Open Source ESP32.
\end{itemize} \\
\hline

\end{tabular}
\end{table}


\subsubsubsection{Software}
\begin{table}[h!]
\centering
\caption{Phân tích rủi ro phần mềm và phương án xử lý}
\label{tab:rui_ro_sw}
\renewcommand{\arraystretch}{1.3} 
\small 

% Giữ nguyên độ rộng cột như bảng Firmware để đồng bộ
\begin{tabular}{|c|>{\raggedright\arraybackslash}p{4.5cm}|c|>{\raggedright\arraybackslash}p{7cm}|}
\hline
\centering\textbf{STT} & \centering\textbf{Rủi ro (Risk)} & \textbf{Mức độ} & \centering\textbf{Phương án giảm thiểu} \tabularnewline
\hline

1 & 
\textbf{Nghẽn cổ chai Database:} \newline Khi hàng trăm thiết bị gửi dữ liệu cùng lúc, việc Insert từng dòng gây quá tải. & 
\centering Cao & 
\begin{itemize}[nosep, leftmargin=*]
    \item Sử dụng kỹ thuật Batch Insert (gom 100-500 bản ghi/lần).
    \item Tận dụng kiến trúc Hypertables của TimescaleDB để phân mảnh dữ liệu.
\end{itemize} \\
\hline

2 & 
\textbf{Mất toàn vẹn dữ liệu (Data Inconsistency):} \newline Lỗi xảy ra giữa chừng khi tạo Nông trại và gán Thiết bị. & 
\centering Cao & 
\begin{itemize}[nosep, leftmargin=*]
    \item Sử dụng Database Transaction (ACID) để đảm bảo "All or Nothing".
    \item Cơ chế Rollback tự động khi có lỗi.
\end{itemize} \\
\hline

3 & 
\textbf{Gián đoạn thời gian thực:} \newline Kết nối WebSocket bị ngắt khiến Dashboard không cập nhật số liệu. & 
\centering Trung bình & 
\begin{itemize}[nosep, leftmargin=*]
    \item Cơ chế Heartbeat (Ping/Pong) để phát hiện mất kết nối.
    \item Tự động kết nối lại (Auto-reconnect) ở phía Frontend React.
\end{itemize} \\
\hline

4 & 
\textbf{Mất dữ liệu tại Broker:} \newline RabbitMQ bị đầy hàng đợi (Queue overflow) nếu Backend xử lý chậm. & 
\centering Trung bình & 
\begin{itemize}[nosep, leftmargin=*]
    \item Cấu hình Message Acknowledgment (chỉ xóa tin khi đã xử lý xong).
    \item Tăng số lượng Worker (Consumer) trong NestJS để xử lý song song.
\end{itemize} \\
\hline

5 & 
\textbf{Dữ liệu rác (Invalid Schema):} \newline Thiết bị gửi sai định dạng JSON làm lỗi parser hệ thống. & 
\centering Cao & 
\begin{itemize}[nosep, leftmargin=*]
    \item Triển khai lớp Validation (DTO) chặt chẽ tại đầu vào (Gateway).
    \item Ghi log gói tin lỗi ra bảng riêng (Dead Letter Queue) để debug.
\end{itemize} \\
\hline

6 & 
\textbf{Sự cố Vận hành (Deployment):} \newline Docker Container bị crash do lỗi bộ nhớ hoặc lỗi runtime. & 
\centering Cao & 
\begin{itemize}[nosep, leftmargin=*]
    \item Cấu hình Docker Restart Policy (\texttt{restart: always}).
    \item Thiết lập Health Check định kỳ cho các service.
\end{itemize} \\
\hline

\end{tabular}
\end{table}
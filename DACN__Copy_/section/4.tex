\newpage
\section{Phân tích yêu cầu}
\subsection{Mục đích}
\subsection{Phương pháp thu thập yêu cầu}
\subsection{Kết quả thu thập yêu cầu}

\subsection{Yêu cầu chức năng}
Dựa trên phân tích hiện trạng và mô hình nghiệp vụ, các yêu cầu chức năng của hệ thống được chia thành các nhóm sau:

\subsubsection*{Nhóm 1: Tổng quan (Dashboard Overview)}
\begin{description}
    \item[FR-03] Hệ thống phải hiển thị \textbf{thông số tổng quan} ngay khi đăng nhập, bao gồm: Tổng số người dùng, Số nông trại đang hoạt động (Active Farms), Số thiết bị đang kết nối (Connected Devices) và Thời gian hoạt động của hệ thống (System Uptime).
    \item[FR-04] Hệ thống phải trực quan hóa dữ liệu hoạt động hàng tuần (Weekly Activity) dưới dạng \textbf{biểu đồ cột hoặc đường}, cho phép Admin so sánh nhanh số lượng người dùng mới, thiết bị mới và bài viết mới.
    \item[FR-05] Hệ thống phải hiển thị danh sách \textbf{Cảnh báo hệ thống (System Alerts)} gần nhất (ví dụ: Nhiệt độ cao, Lịch bảo trì) ngay trên màn hình chính để Admin xử lý kịp thời.
    \item[FR-06] Hệ thống phải cung cấp các nút \textbf{Thao tác nhanh (Quick Actions)} cho phép Admin tạo nhanh Người dùng mới, Nông trại mới hoặc Thêm thiết bị mới chỉ với 1 thao tác.
\end{description}

\subsubsection*{Nhóm 2: Quản lý Nông trại và Người dùng}
\begin{description}
    \item[FR-07 (Quản lý Nông trại)] Hệ thống phải cho phép Admin thực hiện đầy đủ các chức năng (CRUD): Tạo mới, Xem chi tiết, Cập nhật thông tin và Xóa (hoặc vô hiệu hóa) một nông trại khỏi hệ thống.
    \item[FR-08 (Quản lý Người dùng)] Hệ thống phải cho phép quản lý danh sách người dùng (Thương lái/Chủ vườn), bao gồm việc cấp phát tài khoản và phân quyền truy cập (Role-based Access Control) vào từng nông trại cụ thể.
    \item[FR-09 (Gán thiết bị)] Hệ thống phải cho phép Admin thực hiện thao tác \textbf{gán (assign)} một hoặc nhiều thiết bị IoT cụ thể vào một nông trại đã định.
\end{description}

\subsubsection*{Nhóm 3: Quản lý và Giám sát Thiết bị}
\begin{description}
    \item[FR-01 (Phát hiện lỗi)] Hệ thống phải có khả năng \textbf{tự động giám sát} dòng dữ liệu thời gian thực từ các cảm biến và phát hiện các trạng thái bất thường (ví dụ: mất kết nối quá 5 phút, giá trị vượt ngưỡng an toàn, hoặc dữ liệu bị "đóng băng").
    \item[FR-02 (Cảnh báo)] Hệ thống phải \textbf{gửi cảnh báo tức thì} đến người quản trị (thông qua giao diện Dashboard, Email hoặc thông báo đẩy) ngay khi một lỗi cảm biến được xác nhận.
    \item[FR-10 (Đăng ký thiết bị)] Hệ thống phải cho phép đăng ký thiết bị ESP32 mới vào hệ thống thông qua mã định danh duy nhất (Device ID/MAC Address).
    \item[FR-11 (Giám sát trạng thái)] Hệ thống phải hiển thị trạng thái kết nối thời gian thực (Online/Offline) của từng thiết bị. Nếu thiết bị mất kết nối quá thời gian quy định (ví dụ: 5 phút), hệ thống phải tự động cập nhật trạng thái sang Offline.
    \item[FR-12 (Điều khiển từ xa)] Hệ thống phải cho phép người dùng gửi lệnh điều khiển (Bật/Tắt) xuống các thiết bị chấp hành (Actuators) thông qua giao diện Web.
\end{description}

\subsection{Yêu cầu phi chức năng}

\begin{description}
    \item[NFR-01: Hiệu năng và Khả năng chịu tải] \hfill \\
    \begin{itemize}
        \item \textbf{Môi trường triển khai:} Hệ thống phải hoạt động ổn định trên hạ tầng thử nghiệm bao gồm 01 thiết bị phần cứng thực (ESP32) kết hợp với mạng lưới thiết bị mô phỏng (Simulated Devices).
        \item \textbf{Khả năng chịu tải đồng thời:} Hệ thống phải duy trì kết nối và xử lý dữ liệu ổn định từ \textbf{50 thiết bị mô phỏng} gửi dữ liệu liên tục (tần suất 5 giây/bản tin) thông qua giao thức MQTT.
        \item \textbf{Độ trễ xử lý (Latency):} 
            \begin{itemize}
                \item Đối với thiết bị thực: Độ trễ hiển thị dữ liệu lên Dashboard \textbf{dưới 5 giây} (trong điều kiện mạng tiêu chuẩn).
                \item Đối với thiết bị mô phỏng: Đảm bảo không xảy ra hiện tượng mất gói tin (packet loss) tại Message Broker khi tải đạt đỉnh.
            \end{itemize}
        \item \textbf{Thời gian phản hồi Web:} Các thao tác truy xuất dữ liệu lịch sử hoặc tải danh sách thiết bị trên Web Admin phải hoàn tất trong vòng \textbf{dưới 2 giây}.
    \end{itemize}

    \item[NFR-02: Khả năng mở rộng] \hfill \\
    Hệ thống phải hỗ trợ việc thêm mới thiết bị hoặc nông trại mà không cần tắt server để bảo trì (Zero-downtime scaling). Kiến trúc Microservices cho phép mở rộng độc lập module \texttt{IoT Data Processor}.

    \item[NFR-03: Độ tin cậy và Tính sẵn sàng] \hfill \\
    Hệ thống phải đảm bảo thời gian hoạt động (Uptime) đạt \textbf{99\%}. Cơ chế \textbf{Auto-reconnect} phải hoạt động hiệu quả để Dashboard và Thiết bị tự động kết nối lại ngay khi đường truyền internet được khôi phục.

    \item[NFR-04: Bảo mật] \hfill \\
    Dữ liệu truyền từ thiết bị về Server phải được mã hóa hoặc xác thực quyền truy cập. Mật khẩu người dùng phải được băm (hashing) trước khi lưu vào cơ sở dữ liệu.
\end{description}
\subsection{Các mô hình phân tích UML}

\subsubsection{Use case: Giám sát Nông trại}
\begin{figure}[H]
    \centering
    \includegraphics[width=0.9\textwidth]{img/Farm.png}
    \caption{Sơ đồ quản lý farm}
\end{figure}

\begin{table}[H]
    \centering
    \small
    \renewcommand{\arraystretch}{1.3} % Tăng khoảng cách dòng cho dễ đọc
    \begin{tabular}{|p{3.2cm}|p{11.5cm}|}
        \hline
        \textbf{Mã số usecase} & UC-01: Add Farm \\
        \hline
        \textbf{Tên usecase} & Tạo farm \\
        \hline
        \textbf{Mô tả} & Admin tạo 1 Farm \\
        \hline
        \textbf{Actor} & System Admin \\
        \hline
        \textbf{Tiền điều kiện} & Tài khoản Admin đã đăng nhập và có quyền truy cập module Farm. \\
        \hline
        \textbf{Hậu điều kiện} & Farm mới tạo được hiển thị trên danh sách farm \\
        \hline
        \textbf{Trigger} & Admin nhấn chọn menu ``Farm'' trên thanh điều hướng. \\
        \hline
        \textbf{Luồng chính} &
        \begin{enumerate}[leftmargin=*]
            \item Admin truy cập màn hình Farm.
            \item Hệ thống kiểm tra quyền xem của Admin.
            \item Admin chọn nút ``Add Farm''.
            \item Hệ thống hiển thị Form để nhập thông tin.
            \item Admin điền thông tin cho farm rồi ấn nút ''Save''
            \item Hệ thống hiển thị cập nhật Farm mới vào danh sách
        \end{enumerate} \\
        \hline
        \textbf{Quy tắc nghiệp vụ} &
        \begin{itemize}[leftmargin=*]
    
    \item \textbf{BR-01 (Authorization):} Chỉ tài khoản có vai trò \textbf{System Admin} mới được phép tạo Farm.
    \item \textbf{BR-02 (Required Fields):} Các thông tin bắt buộc của Farm (ví dụ: Tên Farm, Vị trí) không được để trống.
    \item \textbf{BR-03 (Unique Farm Name):} Tên Farm phải là duy nhất và không được trùng lặp với các Farm đã tồn tại trong hệ thống.
    \item \textbf{BR-04 (Data Validation):} Dữ liệu nhập vào phải đúng định dạng và nằm trong phạm vi cho phép theo quy định của hệ thống.


        \end{itemize} \\
        \hline
        \textbf{Luồng thay thế / Mở rộng} &
       \begin{itemize}[leftmargin=*]
    \item \textbf{E-01 (Cancel Create Farm):} Tại bước 4--5, Admin nhấn nút mũi tên ``Back'' ở góc trên bên trái hoặc nút ``Cancel'' cạnh nút ``Save'' 
    \textrightarrow{} Hệ thống hủy thao tác tạo Farm và quay về màn hình danh sách Farm, không lưu dữ liệu.
\end{itemize}\\

        \hline
    \end{tabular}
\end{table}

\begin{table}[H]
    \centering
    \small
    \renewcommand{\arraystretch}{1.3}
    \begin{tabular}{|p{3.2cm}|p{11.5cm}|}
        \hline
        \textbf{Mã số usecase} & UC-02: View Farm List \\
        \hline
        \textbf{Tên usecase} & Xem danh sách Farm \\
        \hline
        \textbf{Mô tả} & Admin xem danh sách các Farm có trong hệ thống \\
        \hline
        \textbf{Actor} & System Admin \\
        \hline
        \textbf{Tiền điều kiện} & Tài khoản Admin đã đăng nhập và có quyền truy cập module Farm \\
        \hline
        \textbf{Hậu điều kiện} & Danh sách Farm được hiển thị trên màn hình \\
        \hline
        \textbf{Trigger} & Admin nhấn chọn menu ``Farm'' trên thanh điều hướng \\
        \hline
        \textbf{Luồng chính} &
        \begin{enumerate}[leftmargin=*]
            \item Admin truy cập menu Farm.
            \item Hệ thống kiểm tra quyền truy cập của Admin.
            \item Hệ thống truy vấn dữ liệu Farm từ cơ sở dữ liệu.
            \item Hệ thống hiển thị danh sách Farm.
        \end{enumerate} \\
        \hline
        \textbf{Quy tắc nghiệp vụ} &
        \begin{itemize}[leftmargin=*]
            \item \textbf{BR-01 (Authorization):} Chỉ tài khoản có vai trò \textbf{System Admin} mới được phép xem danh sách Farm.
            \item \textbf{BR-02 (Pagination):} Danh sách Farm được phân trang để đảm bảo hiệu năng hiển thị.
        \end{itemize} \\
        \hline
    \end{tabular}
\end{table}






\begin{table}[H]
    \centering
    \small
    \renewcommand{\arraystretch}{1.3}
    \begin{tabular}{|p{3.2cm}|p{11.5cm}|}
        \hline
        \textbf{Mã số usecase} & UC-05: Edit Farm \\
        \hline
        \textbf{Tên usecase} & Chỉnh sửa Farm \\
        \hline
        \textbf{Mô tả} & Admin chỉnh sửa thông tin Farm \\
        \hline
        \textbf{Actor} & System Admin \\
        \hline
        \textbf{Tiền điều kiện} & Farm tồn tại trong hệ thống \\
        \hline
        \textbf{Hậu điều kiện} & Thông tin Farm được cập nhật \\
        \hline
        \textbf{Trigger} & Admin chọn nút ``Edit'' \\
        \hline
        \textbf{Luồng chính} &
        \begin{enumerate}[leftmargin=*]
            \item Admin chọn Farm cần chỉnh sửa.
            \item Hệ thống hiển thị form chỉnh sửa.
            \item Admin cập nhật thông tin.
            \item Admin nhấn ``Save''.
            \item Hệ thống lưu và cập nhật danh sách Farm.
        \end{enumerate} \\
        \hline
        \textbf{Quy tắc nghiệp vụ} &
        \begin{itemize}[leftmargin=*]
            \item \textbf{BR-01 (Authorization):} Chỉ System Admin được phép chỉnh sửa Farm.
            \item \textbf{BR-02 (Validation):} Dữ liệu chỉnh sửa phải hợp lệ.
        \end{itemize} \\
        \hline
    \end{tabular}
\end{table}

\begin{table}[H]
    \centering
    \small
    \renewcommand{\arraystretch}{1.3}
    \begin{tabular}{|p{3.2cm}|p{11.5cm}|}
        \hline
        \textbf{Mã số usecase} & UC-06: Delete Farm \\
        \hline
        \textbf{Tên usecase} & Xóa Farm \\
        \hline
        \textbf{Mô tả} & Admin xóa Farm khỏi hệ thống \\
        \hline
        \textbf{Actor} & System Admin \\
        \hline
        \textbf{Tiền điều kiện} & Farm tồn tại \\
        \hline
        \textbf{Hậu điều kiện} & Farm bị xóa khỏi danh sách \\
        \hline
        \textbf{Trigger} & Admin chọn nút ``Delete'' \\
        \hline
        \textbf{Luồng chính} &
        \begin{enumerate}[leftmargin=*]
            \item Admin chọn Farm cần xóa.
            \item Hệ thống hiển thị hộp thoại xác nhận.
            \item Admin xác nhận xóa.
            \item Hệ thống xóa Farm và cập nhật danh sách.
        \end{enumerate} \\
        \hline
        \textbf{Quy tắc nghiệp vụ} &
        \begin{itemize}[leftmargin=*]
            \item \textbf{BR-01 (Confirmation):} Phải xác nhận trước khi xóa.
        \end{itemize} \\
        \hline
    \end{tabular}
\end{table}

\begin{table}[H]
    \centering
    \small
    \renewcommand{\arraystretch}{1.3}
    \begin{tabular}{|p{3.2cm}|p{11.5cm}|}
        \hline
        \textbf{Mã số usecase} & UC-07: Import/Export Farm \\
        \hline
        \textbf{Tên usecase} & Import / Export Farm \\
        \hline
        \textbf{Mô tả} & Admin nhập hoặc xuất dữ liệu Farm \\
        \hline
        \textbf{Actor} & System Admin \\
        \hline
        \textbf{Tiền điều kiện} & Admin có quyền truy cập module Farm \\
        \hline
        \textbf{Hậu điều kiện} & Dữ liệu Farm được nhập hoặc xuất thành công \\
        \hline
        \textbf{Trigger} & Admin chọn chức năng Import hoặc Export \\
        \hline
        \textbf{Luồng chính} &
        \begin{enumerate}[leftmargin=*]
            \item Admin chọn Import hoặc Export.
            \item Hệ thống hiển thị tùy chọn file.
            \item Admin xác nhận thao tác.
            \item Hệ thống xử lý dữ liệu.
        \end{enumerate} \\
        \hline
        \textbf{Quy tắc nghiệp vụ} &
        \begin{itemize}[leftmargin=*]
            \item \textbf{BR-01 (File Format):} Chỉ hỗ trợ định dạng cho phép (CSV, Excel).
        \end{itemize} \\
        \hline
    \end{tabular}
\end{table}

\begin{table}[H]
    \centering
    \small
    \renewcommand{\arraystretch}{1.3}
    \begin{tabular}{|p{3.2cm}|p{11.5cm}|}
        \hline
        \textbf{Mã số usecase} & UC-08: View Farm Detail \\
        \hline
        \textbf{Tên usecase} & Xem chi tiết Farm \\
        \hline
        \textbf{Mô tả} & Admin xem thông tin chi tiết của một Farm trong hệ thống \\
        \hline
        \textbf{Actor} & System Admin \\
        \hline
        \textbf{Tiền điều kiện} & 
        \begin{itemize}[leftmargin=*]
            \item Admin đã đăng nhập
            \item Danh sách Farm đang được hiển thị
        \end{itemize} \\
        \hline
        \textbf{Hậu điều kiện} & Thông tin chi tiết của Farm được hiển thị \\
        \hline
        \textbf{Trigger} & Admin nhấn chọn một Farm trong danh sách \\
        \hline
        \textbf{Luồng chính} &
        \begin{enumerate}[leftmargin=*]
            \item Admin xem danh sách Farm.
            \item Admin nhấn chọn một Farm bất kỳ.
            \item Hệ thống kiểm tra quyền truy cập của Admin.
            \item Hệ thống truy vấn thông tin chi tiết của Farm.
            \item Hệ thống hiển thị màn hình chi tiết Farm.
        \end{enumerate} \\
        \hline
        \textbf{Quy tắc nghiệp vụ} &
        \begin{itemize}[leftmargin=*]
            \item \textbf{BR-01 (Authorization):} Chỉ tài khoản có vai trò \textbf{System Admin} mới được phép xem chi tiết Farm.
            \item \textbf{BR-02 (Data Integrity):} Farm phải tồn tại trong hệ thống tại thời điểm truy vấn.
        \end{itemize} \\
        \hline
        \textbf{Luồng thay thế / Mở rộng} &
        \begin{itemize}[leftmargin=*]
           
            \item \textbf{E-01 (Back to List):} Tại màn hình chi tiết, Admin nhấn nút mũi tên ``Back'' ở góc trên bên trái  
            \textrightarrow{} Hệ thống quay về màn hình danh sách Farm.
        \end{itemize} \\
        \hline
    \end{tabular}
\end{table}

\newpage

\subsubsection{Use case: Giám sát Thiết bị}
\begin{figure}[H]
    \centering
    \begin{subfigure}{0.95\textwidth}
        \centering
        \includegraphics[width=0.95\textwidth]{img/Device.png}
        \caption{Sơ đồ quản lý thiết bị}
    \end{subfigure}
\end{figure}


\begin{table}[H]
    \centering
    \small
    \renewcommand{\arraystretch}{1.3}
    \begin{tabular}{|p{3.2cm}|p{11.5cm}|}
        \hline
        \textbf{Mã số usecase} & UC-01: View device list \\
        \hline
        \textbf{Tên usecase} & Xem danh sách thiết bị \\
        \hline
        \textbf{Mô tả} & Admin xem danh sách các thiết bị trong hệ thống \\
        \hline
        \textbf{Actor} & System Admin \\
        \hline
        \textbf{Tiền điều kiện} & Admin đã đăng nhập và có quyền truy cập module Device \\
        \hline
        \textbf{Hậu điều kiện} & Danh sách thiết bị được hiển thị \\
        \hline
        \textbf{Trigger} & Admin chọn menu ``Device'' \\
        \hline
        \textbf{Luồng chính} &
        \begin{enumerate}[leftmargin=*]
            \item Admin truy cập module Device.
            \item Hệ thống kiểm tra quyền truy cập.
            \item Hệ thống hiển thị danh sách thiết bị.
        \end{enumerate} \\
        \hline
        \textbf{Quy tắc nghiệp vụ} &
        \begin{itemize}[leftmargin=*]
            \item \textbf{BR-01 (Authorization):} Chỉ System Admin được phép xem danh sách thiết bị.
        \end{itemize} \\
        \hline
    \end{tabular}
\end{table}
\begin{table}[H]
    \centering
    \small
    \renewcommand{\arraystretch}{1.3}
    \begin{tabular}{|p{3.2cm}|p{11.5cm}|}
        \hline
        \textbf{Mã số usecase} & UC-02: View device detail \\
        \hline
        \textbf{Tên usecase} & Xem chi tiết thiết bị \\
        \hline
        \textbf{Mô tả} & Admin xem thông tin chi tiết của một thiết bị \\
        \hline
        \textbf{Actor} & System Admin \\
        \hline
        \textbf{Tiền điều kiện} & Danh sách thiết bị đã được hiển thị \\
        \hline
        \textbf{Hậu điều kiện} & Thông tin chi tiết thiết bị được hiển thị \\
        \hline
        \textbf{Trigger} & Admin chọn một thiết bị trong danh sách \\
        \hline
        \textbf{Luồng chính} &
        \begin{enumerate}[leftmargin=*]
            \item Admin nhấn chọn một thiết bị.
            \item Hệ thống tải dữ liệu chi tiết thiết bị.
            \item Hệ thống hiển thị màn hình chi tiết thiết bị.
        \end{enumerate} \\
        \hline
        \textbf{Luồng thay thế / Mở rộng} &
        \begin{itemize}[leftmargin=*]
            \item \textbf{E-01 (Back):} Admin nhấn nút ``Back'' \textrightarrow{} hệ thống quay lại danh sách thiết bị.
        \end{itemize} \\
        \hline
    \end{tabular}
\end{table}
\begin{table}[H]
    \centering
    \small
    \renewcommand{\arraystretch}{1.3}
    \begin{tabular}{|p{3.2cm}|p{11.5cm}|}
        \hline
        \textbf{Mã số usecase} & UC-03: Add device \\
        \hline
        \textbf{Tên usecase} & Thêm thiết bị \\
        \hline
        \textbf{Mô tả} & Admin tạo mới một thiết bị \\
        \hline
        \textbf{Actor} & System Admin \\
        \hline
        \textbf{Tiền điều kiện} & Admin có quyền quản lý thiết bị \\
        \hline
        \textbf{Hậu điều kiện} & Thiết bị mới được lưu và hiển thị trong danh sách \\
        \hline
        \textbf{Trigger} & Admin nhấn nút ``Add Device'' \\
        \hline
        \textbf{Luồng chính} &
        \begin{enumerate}[leftmargin=*]
            \item Admin mở màn hình Device.
            \item Admin chọn ``Add Device''.
            \item Hệ thống hiển thị form nhập thông tin.
            \item Admin nhập thông tin và nhấn ``Save''.
            \item Hệ thống lưu thiết bị và cập nhật danh sách.
        \end{enumerate} \\
        \hline
        \textbf{Quy tắc nghiệp vụ} &
        \begin{itemize}[leftmargin=*]
            \item \textbf{BR-01:} Các trường bắt buộc không được để trống.
      
        \end{itemize} \\
        \hline
    \end{tabular}
\end{table}
\begin{table}[H]
    \centering
    \small
    \renewcommand{\arraystretch}{1.3}
    \begin{tabular}{|p{3.2cm}|p{11.5cm}|}
        \hline
        \textbf{Mã số usecase} & UC-04: Edit device \\
        \hline
        \textbf{Tên usecase} & Chỉnh sửa thiết bị \\
        \hline
        \textbf{Mô tả} & Admin cập nhật thông tin thiết bị \\
        \hline
        \textbf{Actor} & System Admin \\
        \hline
        \textbf{Tiền điều kiện} & Thiết bị đã tồn tại \\
        \hline
        \textbf{Hậu điều kiện} & Thông tin thiết bị được cập nhật \\
        \hline
        \textbf{Trigger} & Admin chọn ``Edit'' tại thiết bị \\
        \hline
        \textbf{Luồng chính} &
        \begin{enumerate}[leftmargin=*]
            \item Admin chọn thiết bị cần chỉnh sửa.
            \item Hệ thống hiển thị form chỉnh sửa.
            \item Admin cập nhật thông tin và nhấn ``Save''.
            \item Hệ thống lưu thay đổi.
        \end{enumerate} \\
        \hline
      
\textbf{Quy tắc nghiệp vụ} &
\begin{itemize}[leftmargin=*]
    \item \textbf{BR-01 (Authorization):} Chỉ tài khoản có vai trò \textbf{System Admin} mới được phép chỉnh sửa thiết bị.
    \item \textbf{BR-02 (Data Integrity):} Thiết bị phải tồn tại trong hệ thống tại thời điểm chỉnh sửa.
    \item \textbf{BR-03 (Data Validation):} Dữ liệu chỉnh sửa phải đúng định dạng và nằm trong phạm vi cho phép của hệ thống.
\end{itemize} \\

\hline
\textbf{Luồng thay thế / Mở rộng} &
\begin{itemize}[leftmargin=*]
  
    \item \textbf{E-01 (Cancel Edit):} Tại bước 3--4, Admin nhấn nút ``Cancel'' hoặc ``Back'' \textrightarrow{} Hệ thống hủy thao tác chỉnh sửa và không lưu dữ liệu.
\end{itemize} \\
\hline

    \end{tabular}
\end{table}
\begin{table}[H]
    \centering
    \small
    \renewcommand{\arraystretch}{1.3}
    \begin{tabular}{|p{3.2cm}|p{11.5cm}|}
        \hline
        \textbf{Mã số usecase} & UC-05: Delete device \\
        \hline
        \textbf{Tên usecase} & Xóa thiết bị \\
        \hline
        \textbf{Mô tả} & Admin xóa thiết bị khỏi hệ thống \\
        \hline
        \textbf{Actor} & System Admin \\
        \hline
        \textbf{Tiền điều kiện} & Thiết bị tồn tại trong hệ thống \\
        \hline
        \textbf{Hậu điều kiện} & Thiết bị bị xóa khỏi danh sách \\
        \hline
        \textbf{Trigger} & Admin chọn ``Delete'' tại thiết bị \\
        \hline
        \textbf{Luồng chính} &
        \begin{enumerate}[leftmargin=*]
            \item Admin chọn thiết bị cần xóa.
            \item Hệ thống hiển thị hộp thoại xác nhận.
            \item Admin xác nhận xóa.
            \item Hệ thống xóa thiết bị.
        \end{enumerate} \\
        \hline
   
\textbf{Quy tắc nghiệp vụ} &
\begin{itemize}[leftmargin=*]
    \item \textbf{BR-01 (Authorization):} Chỉ tài khoản có vai trò \textbf{System Admin} mới được phép xóa thiết bị.
    \item \textbf{BR-02 (Data Integrity):} Thiết bị phải tồn tại trong hệ thống tại thời điểm xóa.
   
\end{itemize} \\

\hline
\textbf{Luồng thay thế / Mở rộng} &
\begin{itemize}[leftmargin=*]
    \item \textbf{E-01 (Cancel Delete):} Tại bước xác nhận, Admin chọn ``Cancel'' \textrightarrow{} Hệ thống hủy thao tác xóa thiết bị.
    \item \textbf{E-02 (Delete Not Allowed):} Nếu thiết bị còn sensor liên kết \textrightarrow{} Hệ thống hiển thị thông báo không thể xóa thiết bị.
\end{itemize} \\
\hline

    \end{tabular}
\end{table}
\begin{table}[H]
    \centering
    \small
    \renewcommand{\arraystretch}{1.3}
    \begin{tabular}{|p{3.2cm}|p{11.5cm}|}
        \hline
        \textbf{Mã số usecase} & UC-06: Import/Export device \\
        \hline
        \textbf{Tên usecase} & Import / Export thiết bị \\
        \hline
        \textbf{Mô tả} & Admin nhập hoặc xuất danh sách thiết bị \\
        \hline
        \textbf{Actor} & System Admin \\
        \hline
        \textbf{Tiền điều kiện} & Admin có quyền quản lý thiết bị \\
        \hline
        \textbf{Hậu điều kiện} & Dữ liệu thiết bị được nhập hoặc xuất thành công \\
        \hline
        \textbf{Trigger} & Admin chọn ``Import/Export Device'' \\
        \hline
        \textbf{Luồng chính} &
        \begin{enumerate}[leftmargin=*]
            \item Admin chọn chức năng Import hoặc Export.
            \item Hệ thống xử lý dữ liệu.
            \item Hệ thống thông báo kết quả.
        \end{enumerate} \\
        \hline
    \end{tabular}
\end{table}

\subsubsection{Use case: Monitor System Analytics}
\begin{figure}[H]
    \centering
    \includegraphics[width=0.9\textwidth]{img/analytics.png}
    \caption{Sơ đồ luồng quản lý và tạo báo cáo}
\end{figure}
\begin{table}[H]
    \centering
    \small
    \renewcommand{\arraystretch}{1.3} % Tăng khoảng cách dòng cho dễ đọc
    \begin{tabular}{|p{3.2cm}|p{11.5cm}|}
        \hline
        \textbf{Mã số usecase} & UC-01: Monitor System Analytics \\
        \hline
        \textbf{Tên usecase} & Giám sát Phân tích Hệ thống \\
        \hline
        \textbf{Mô tả} & Admin theo dõi tổng quan các chỉ số hoạt động, xếp hạng hiệu suất nông trại và thống kê thiết bị trên Dashboard. \\
        \hline
        \textbf{Actor} & System Admin \\
        \hline
        \textbf{Tiền điều kiện} & Tài khoản Admin đã đăng nhập và có quyền truy cập module Analytics. \\
        \hline
        \textbf{Hậu điều kiện} & Dữ liệu thống kê được hiển thị đầy đủ và cập nhật mới nhất. \\
        \hline
        \textbf{Trigger} & Admin nhấn chọn menu ``Analytics \& Insights'' trên thanh điều hướng. \\
        \hline
        \textbf{Luồng chính} &
        \begin{enumerate}[leftmargin=*]
            \item Admin truy cập màn hình Analytics.
            \item Hệ thống kiểm tra quyền xem của Admin.
            \item Hệ thống tải dữ liệu tổng hợp từ cơ sở dữ liệu.
            \item Hệ thống hiển thị Dashboard Metrics (doanh thu, sản lượng, nhiệt độ trung bình).
            \item Hệ thống hiển thị Farm Performance Rankings (xếp hạng hiệu suất).
            \item Hệ thống hiển thị Device Statistics (biểu đồ trạng thái Online/Offline).
        \end{enumerate} \\
        \hline
        \textbf{Quy tắc nghiệp vụ} &
        \begin{itemize}[leftmargin=*]
            \item \textbf{BR-01 (Data Latency):} Dữ liệu hiển thị trên Dashboard phải được cập nhật gần thời gian thực (Real-time), độ trễ tối đa không quá 30 giây.
            \item \textbf{BR-02 (Default View):} Mặc định hiển thị dữ liệu tổng hợp của toàn hệ thống trong 7 ngày gần nhất.
            \item \textbf{BR-03 (Access Control):} Chỉ tài khoản Admin có quyền ``Manage'' mới hiển thị nút cấu hình ``Configure Alert Thresholds''.
            \item \textbf{BR-04 (Ranking Logic):} Xếp hạng nông trại dựa trên chỉ số KPI tổng hợp (tỷ lệ sản lượng / mức tiêu thụ năng lượng).
        \end{itemize} \\
        \hline
        \textbf{Luồng thay thế / Mở rộng} &
        \begin{itemize}[leftmargin=*]
            \item \textbf{E-01 (Filter):} Tại bước 4-6, Admin chọn bộ lọc thời gian hoặc khu vực \textrightarrow{} Hệ thống truy vấn lại và cập nhật hiển thị.
            \item \textbf{E-02 (Export):} Admin bấm ``Export'' \textrightarrow{} Hệ thống kiểm tra định dạng file (PDF/CSV) và tiến hành tải xuống.
            \item \textbf{E-03 (Alert Config):} Admin thay đổi ngưỡng cảnh báo \textrightarrow{} Hệ thống lưu quy tắc mới vào cơ sở dữ liệu và áp dụng ngay lập tức.
        \end{itemize} \\
        \hline
    \end{tabular}
\end{table}

\newpage

\subsubsection{Use case: Troubleshoot System}
\begin{figure}[H]
    \centering
    \begin{subfigure}{0.95\textwidth}
        \centering
        \includegraphics[width=0.95\textwidth]{img/troubleshoot.png}
        \caption{Sơ đồ xử lý sự cố và theo dõi tình trạng hệ thống}
    \end{subfigure}
\end{figure}
\noindent Hình mô tả lần lượt quá trình kiểm tra dịch vụ, khắc phục từ xa, và cách truy xuất log, xem lịch sử cảnh báo để đối soát.

\begin{table}[H]
    \centering
    \small
    \renewcommand{\arraystretch}{1.3}
    \begin{tabular}{|p{3.2cm}|p{11.5cm}|}
        \hline
        \textbf{Mã số usecase} & UC-02: Troubleshoot System \\
        \hline
        \textbf{Tên usecase} & Kiểm tra và xử lý sự cố \\
        \hline
        \textbf{Mô tả} & Quy trình xem tình trạng sức khỏe hệ thống, đọc log chi tiết và gửi lệnh khắc phục sự cố từ xa. \\
        \hline
        \textbf{Actor} & System Admin \\
        \hline
        \textbf{Tiền điều kiện} & Hệ thống phát hiện sự cố (Alert) hoặc Admin thực hiện kiểm tra định kỳ. \\
        \hline
        \textbf{Hậu điều kiện} & Nguyên nhân lỗi được xác định hoặc lệnh sửa chữa đã được gửi đi. \\
        \hline
        \textbf{Trigger} & Admin nhận được thông báo lỗi hoặc truy cập menu ``Troubleshoot''. \\
        \hline
        \textbf{Luồng chính} &
        \begin{enumerate}[leftmargin=*]
            \item Admin truy cập trang Troubleshoot.
            \item Hệ thống hiển thị đèn trạng thái (Health Indicators) của Server, API và Mạng.
            \item Hệ thống truy xuất và hiển thị danh sách 50 dòng System Logs gần nhất.
            \item Admin phân tích các chỉ số màu đỏ (Lỗi) và nội dung log để xác định nguyên nhân.
        \end{enumerate} \\
        \hline
        \textbf{Quy tắc nghiệp vụ} &
        \begin{itemize}[leftmargin=*]
            \item \textbf{BR-05 (Health Coding):} Trạng thái hệ thống phải được mã hóa màu: Xanh (Ổn định), Vàng (Cảnh báo tải >80\%), Đỏ (Lỗi/Ngừng hoạt động).
            \item \textbf{BR-06 (Log Privacy):} Các thông tin nhạy cảm trong log (như Password, API Key) phải được che dấu (masking) bằng ký tự `******' trước khi hiển thị.
            \item \textbf{BR-07 (Action Permission):} Chỉ tài khoản Super Admin mới có quyền thực thi các lệnh tác động hệ thống như ``Restart Service''.
        \end{itemize} \\
        \hline
        \textbf{Luồng thay thế / Mở rộng} &
        \begin{itemize}[leftmargin=*]
            \item \textbf{E-01 (Perform Diagnostics):} Admin chọn thiết bị lỗi \textrightarrow{} chọn hành động (Ping/Restart Service/Telemetry) \textrightarrow{} Hệ thống thực thi lệnh và trả về kết quả (Success/Timeout).
            \item \textbf{E-02 (Export Logs):} Admin bấm ``Export Logs'' \textrightarrow{} Hệ thống tổng hợp log lỗi thành file CSV/JSON và kích hoạt tải xuống.
            \item \textbf{E-03 (Filter Logs):} Admin lọc theo mức độ ``Error'' \textrightarrow{} Hệ thống ẩn các log thông thường, chỉ hiện log lỗi.
        \end{itemize} \\
        \hline
    \end{tabular}
\end{table}

\newpage

\subsubsection{Use case: Manage Reports}
\begin{figure}[H]
    \centering
    \includegraphics[width=0.9\textwidth]{img/reports.png}
    \caption{Sơ đồ luồng quản lý và tạo báo cáo}
\end{figure}
\noindent Hình thể hiện quy trình xem danh sách báo cáo, tạo mới, lập lịch hoặc tạo báo cáo tùy chỉnh và cập nhật lại danh sách.

\begin{table}[H]
    \centering
    \small
    \renewcommand{\arraystretch}{1.3}
    \begin{tabular}{|p{3.2cm}|p{11.5cm}|}
        \hline
        \textbf{Mã số usecase} & UC-03: Manage Reports \\
        \hline
        \textbf{Tên usecase} & Quản lý và tạo báo cáo \\
        \hline
        \textbf{Mô tả} & Admin xem danh sách báo cáo đã lưu, tạo báo cáo mới tức thì hoặc lập lịch gửi báo cáo tự động. \\
        \hline
        \textbf{Actor} & System Admin \\
        \hline
        \textbf{Tiền điều kiện} & Admin đã đăng nhập thành công. \\
        \hline
        \textbf{Hậu điều kiện} & Danh sách báo cáo được cập nhật; file báo cáo được tạo ra. \\
        \hline
        \textbf{Trigger} & Admin truy cập trang ``Reports''. \\
        \hline
        \textbf{Luồng chính} &
        \begin{enumerate}[leftmargin=*]
            \item Admin mở trang Reports.
            \item Hệ thống truy vấn cơ sở dữ liệu báo cáo.
            \item Hệ thống hiển thị danh sách Recent Reports (bao gồm: Tên, Ngày tạo, Người tạo, Loại báo cáo).
            \item Admin xem thông tin hoặc tải về các báo cáo cũ.
        \end{enumerate} \\
        \hline
        \textbf{Quy tắc nghiệp vụ} &
        \begin{itemize}[leftmargin=*]
            \item \textbf{BR-08 (Retention Policy):} Báo cáo chỉ được lưu trữ trực tuyến trong 90 ngày. Sau thời gian này, dữ liệu sẽ được chuyển sang lưu trữ lạnh (Archive).
            \item \textbf{BR-09 (File Format):} Hệ thống phải hỗ trợ xuất báo cáo ra hai định dạng: PDF (để in ấn/trình ký) và Excel/CSV (để phân tích số liệu).
            \item \textbf{BR-10 (Schedule Limit):} Mỗi tài khoản Admin chỉ được thiết lập tối đa 5 lịch báo cáo tự động để đảm bảo hiệu năng hệ thống.
        \end{itemize} \\
        \hline
        \textbf{Luồng thay thế / Mở rộng} &
        \begin{itemize}[leftmargin=*]
            \item \textbf{E-01 (Generate):} Admin bấm ``Generate New'' \textrightarrow{} Hệ thống thu thập dữ liệu hiện tại và tạo file báo cáo ngay lập tức.
            \item \textbf{E-02 (Schedule):} Admin bấm ``Schedule'' \textrightarrow{} Hệ thống hiển thị form chọn tần suất (hàng ngày/tuần) \textrightarrow{} Lưu lịch chạy tự động (Cron job).
            \item \textbf{E-03 (Custom Report):} Admin chọn các trường dữ liệu tùy chỉnh \textrightarrow{} Bấm ``Create'' \textrightarrow{} Hệ thống tạo báo cáo theo mẫu riêng.
        \end{itemize} \\
        \hline
    \end{tabular}
\end{table}
\input{section/4_quy}
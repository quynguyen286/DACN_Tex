% \section{Cơ sở lý thuyết}
% \subsection{Kiến trúc hệ thống IoT}

% \subsubsection{Mô hình kiến trúc phân lớp (Layered Architecture)}
% Hệ thống được thiết kế dựa trên mô hình kiến trúc phân lớp (Layered Architecture) kết hợp với hướng sự kiện (Event-driven). Việc phân chia này giúp tách biệt các mối quan tâm (Separation of Concerns), trong đó việc thu thập dữ liệu từ thiết bị và việc xử lý logic nghiệp vụ được độc lập với nhau \cite{iot_arch_2015}. Kiến trúc bao gồm các tầng chính:
% \begin{itemize}
%     \item \textbf{Physical Layer:} Các thiết bị ESP32 thu thập dữ liệu môi trường.
%     \item \textbf{Integration Layer:} Đóng vai trò trung gian tiếp nhận dữ liệu qua RabbitMQ và điều phối API qua Gateway.
%     \item \textbf{Business Layer:} Nơi xử lý logic chính, được xây dựng trên nền tảng NestJS.
%     \item \textbf{Database Layer:} Lưu trữ dữ liệu quan hệ và chuỗi thời gian.
% \end{itemize}

% \subsubsection{Tầng Xử lý nghiệp vụ với NestJS}
% Đối với tầng Business Layer, nhóm nghiên cứu lựa chọn \textbf{NestJS} - một framework mã nguồn mở dành cho việc xây dựng các ứng dụng phía máy chủ (server-side) hiệu quả và dễ mở rộng trên nền tảng Node.js \cite{nestjs_docs}.

% Các lý do chính cho việc lựa chọn NestJS trong đề tài này bao gồm:

% \begin{enumerate}
%     \item \textbf{Hỗ trợ kiến trúc Microservices và Giao tiếp bất đồng bộ:} 
%     Đây là yếu tố quan trọng nhất. Khác với các framework truyền thống, NestJS cung cấp cơ chế tích hợp sẵn (out-of-the-box) với các Message Broker như \textbf{RabbitMQ}. Điều này cho phép Backend dễ dàng chuyển đổi sang mô hình hướng sự kiện (Event-driven), giúp hệ thống có thể xử lý hàng nghìn thông điệp từ cảm biến gửi về đồng thời mà không bị tắc nghẽn (Non-blocking I/O) \cite{nestjs_docs}.
    
%     \item \textbf{Kiến trúc Module hóa (Modularity):}
%     NestJS tổ chức mã nguồn thành các Module riêng biệt (như \textit{AuthModule}, \textit{FarmModule}, \textit{DeviceModule}). Cấu trúc này rất phù hợp với bài toán quản lý đa nông trại, giúp nhóm phát triển dễ dàng bảo trì, mở rộng tính năng cho từng phân hệ mà không ảnh hưởng đến toàn bộ hệ thống.
    
%     \item \textbf{Sử dụng TypeScript:}
%     Việc sử dụng TypeScript giúp đảm bảo tính chặt chẽ của dữ liệu (Type safety) \cite{typescript_lang}. Trong các hệ thống IoT, việc định nghĩa chính xác kiểu dữ liệu (Data Payload) từ cảm biến là rất quan trọng để tránh các lỗi logic tiềm ẩn trong quá trình xử lý.
% \end{enumerate}

% \subsection{Các giao thức truyền thông}

% Trong hệ thống IoT Nông nghiệp thông minh, việc lựa chọn giao thức truyền thông phù hợp đóng vai trò quyết định đến độ ổn định và hiệu năng của hệ thống. Nhóm nghiên cứu sử dụng mô hình lai (Hybrid) kết hợp giữa MQTT và HTTP.

% \subsubsection{Giao thức MQTT (Message Queuing Telemetry Transport)}
% MQTT là giao thức truyền thông điệp theo mô hình Publish/Subscribe, được thiết kế chuyên biệt cho các thiết bị có tài nguyên hạn chế và đường truyền mạng không ổn định \cite{mqtt_spec}. 

% Lý do nhóm nghiên cứu chọn MQTT làm giao thức chính để giao tiếp giữa thiết bị Node (ESP32) và Server bao gồm:
% \begin{itemize}
%     \item \textbf{Tối ưu băng thông và năng lượng:} Header của gói tin MQTT rất nhỏ (chỉ từ 2 bytes), giúp giảm tải cho hạ tầng mạng 3G/4G tại nông trại và tiết kiệm năng lượng cho thiết bị chạy pin.
%     \item \textbf{Cơ chế QoS và Keep-alive:} Trong môi trường nông nghiệp rộng lớn, kết nối mạng thường xuyên bị gián đoạn. Cơ chế Keep-alive giúp phát hiện ngay lập tức khi thiết bị mất kết nối (Offline). Đồng thời, các mức QoS (0, 1, 2) đảm bảo dữ liệu quan trọng không bị thất lạc ngay cả khi mạng chập chờn.
%     \item \textbf{Mô hình Publish/Subscribe:} Giúp tách biệt hoàn toàn (Decoupling) giữa thiết bị gửi dữ liệu và hệ thống xử lý, cho phép mở rộng hàng nghìn thiết bị mà không cần sửa đổi mã nguồn phía Server.
% \end{itemize}

% \subsubsection{Giao thức HTTP (Hypertext Transfer Protocol)}
% Mặc dù MQTT rất hiệu quả cho các bản tin ngắn (telemetry), nhưng nó không phù hợp để truyền tải các dữ liệu kích thước lớn (Blob data). Do đó, giao thức HTTP được sử dụng song song cho hai mục đích chính \cite{http_rfc}:

% \begin{enumerate}
%     \item \textbf{Giao tiếp giữa Web Admin và Backend:} Hệ thống sử dụng kiến trúc RESTful API dựa trên HTTP để thực hiện các tác vụ quản lý (CRUD) từ phía người dùng, đảm bảo tính tương thích cao với các trình duyệt web và thiết bị di động.
%     \item \textbf{Truyền tải hình ảnh từ thiết bị:} Đối với tính năng giám sát hình ảnh cây trồng hoặc phát hiện lỗi qua Camera, thiết bị ESP32 sẽ sử dụng phương thức HTTP POST (multipart/form-data) để gửi file ảnh trực tiếp về Server. Cách tiếp cận này giúp tránh gây tắc nghẽn cho MQTT Broker và tận dụng được tốc độ truyền tải file tốt hơn của HTTP.
% \end{enumerate}

% \subsection{Công nghệ Message Broker (RabbitMQ)}

% Trong kiến trúc hệ thống phân tán, RabbitMQ đóng vai trò là xương sống (backbone) cho việc truyền tải dữ liệu, đảm bảo tính toàn vẹn và khả năng mở rộng của hệ thống. Nhóm nghiên cứu lựa chọn RabbitMQ dựa trên ba đặc tính kỹ thuật quan trọng sau \cite{rabbitmq_book}:

% \subsubsection{Cơ chế định tuyến linh hoạt (Flexible Routing)}
% Đây là tính năng then chốt để giải quyết bài toán quản lý đa nông trại. RabbitMQ sử dụng khái niệm \textbf{Topic Exchange}, cho phép định tuyến thông điệp dựa trên các khóa định tuyến (Routing Keys) có cấu trúc phân cấp (ví dụ: \texttt{farm.01.sensor.temp}). 
% \begin{itemize}
%     \item Điều này cho phép Backend (NestJS) có thể đăng ký nhận dữ liệu một cách chọn lọc: chỉ nhận dữ liệu nhiệt độ, hoặc chỉ nhận dữ liệu của một nông trại cụ thể để xử lý, giúp tối ưu hóa luồng dữ liệu và giảm tải xử lý dư thừa.
% \end{itemize}

% \subsubsection{Khả năng chịu tải và Cân bằng tải (Load Leveling)}
% Hệ thống IoT thường đối mặt với thách thức về lưu lượng dữ liệu tăng đột biến (Traffic spikes). RabbitMQ hoạt động như một bộ đệm (Buffer) khổng lồ:
% \begin{itemize}
%     \item Khi hàng nghìn thiết bị ESP32 gửi dữ liệu cùng lúc, RabbitMQ sẽ xếp hàng các thông điệp này vào hàng đợi (Queue) thay vì ép buộc Server xử lý ngay lập tức.
%     \item Server (Consumer) có thể lấy dữ liệu từ hàng đợi theo tốc độ xử lý của riêng mình, đảm bảo hệ thống không bao giờ bị quá tải (Overload) hay mất mát dữ liệu do nghẽn cổ chai.
% \end{itemize}

% \subsubsection{Khả năng tương tác đa giao thức (Protocol Interoperability)}
% RabbitMQ cung cấp khả năng chuyển đổi linh hoạt giữa các giao thức truyền thông. 
% \begin{itemize}
%     \item Trong hệ thống này, RabbitMQ đóng vai trò là "cầu nối": nó tiếp nhận dữ liệu từ thiết bị IoT qua giao thức \textbf{MQTT} (nhẹ, tiết kiệm băng thông), sau đó chuyển đổi và phân phối đến Backend Service qua giao thức \textbf{AMQP 0-9-1} (mạnh mẽ, tin cậy cao).
%     \item Kiến trúc lai này giúp tận dụng được ưu điểm của cả hai thế giới: sự gọn nhẹ cho thiết bị nhúng và sự tin cậy cho hệ thống máy chủ.
% \end{itemize}

% \subsection{Hệ quản trị Cơ sở dữ liệu quan hệ (PostgreSQL)}

% Trước khi áp dụng các công nghệ chuyên biệt cho dữ liệu chuỗi thời gian, hệ thống cần một nền tảng vững chắc để quản lý các dữ liệu có cấu trúc (Structured Data) như thông tin người dùng, danh sách nông trại, và quyền truy cập. Nhóm nghiên cứu lựa chọn \textbf{PostgreSQL} vì các đặc tính kỹ thuật vượt trội sau \cite{postgres_docs}:

% \subsubsection{Tuân thủ chuẩn ACID và Độ tin cậy cao}
% Đối với các phân hệ quản lý (User \& Farm Management), tính toàn vẹn dữ liệu là ưu tiên hàng đầu. PostgreSQL tuân thủ nghiêm ngặt chuẩn ACID (Atomicity, Consistency, Isolation, Durability), đảm bảo các giao dịch quan trọng (như tạo mới nông trại, phân quyền người dùng) luôn được thực hiện chính xác và an toàn, ngay cả khi hệ thống gặp sự cố bất ngờ.

% \subsubsection{Hỗ trợ dữ liệu bán cấu trúc (JSONB)}
% Trong các hệ thống IoT, cấu hình của thiết bị (Device Configurations) thường đa dạng và thay đổi tùy theo loại cảm biến. PostgreSQL cung cấp kiểu dữ liệu \textbf{JSONB}, cho phép lưu trữ các thuộc tính động này một cách linh hoạt mà không cần thay đổi cấu trúc bảng (Schema migration). Điều này giúp hệ thống tận dụng được sự linh hoạt của NoSQL (như MongoDB) ngay bên trong một cơ sở dữ liệu quan hệ mạnh mẽ.

% \subsubsection{Khả năng mở rộng (Extensibility)}
% Khác với các hệ quản trị CSDL khác, PostgreSQL được thiết kế để dễ dàng mở rộng thông qua các \textit{Extensions}. Đây chính là tiền đề kỹ thuật quan trọng để nhóm nghiên cứu có thể tích hợp \textbf{TimescaleDB} (một Extension của PostgreSQL) vào hệ thống, biến PostgreSQL thành một cơ sở dữ liệu lai (Hybrid) mạnh mẽ: vừa xử lý tốt dữ liệu quan hệ, vừa tối ưu cho dữ liệu chuỗi thời gian.

% \subsection{Cơ sở dữ liệu chuỗi thời gian (Time-series Database)}

% Đặc thù của hệ thống giám sát nông nghiệp là việc thu thập dữ liệu môi trường (nhiệt độ, độ ẩm, ánh sáng) diễn ra liên tục theo thời gian thực. Dữ liệu này có tính chất "chỉ thêm vào" (append-only) và khối lượng tăng trưởng rất nhanh theo thời gian \cite{tsdb_importance}. Các hệ quản trị cơ sở dữ liệu quan hệ (RDBMS) truyền thống thường gặp vấn đề về hiệu năng truy vấn (Query performance) khi bảng dữ liệu đạt kích thước hàng triệu bản ghi.

% Để giải quyết vấn đề này, nhóm nghiên cứu lựa chọn \textbf{TimescaleDB} - một cơ sở dữ liệu chuỗi thời gian mã nguồn mở được xây dựng dựa trên PostgreSQL. Các ưu điểm kỹ thuật nổi bật bao gồm \cite{timescaledb_docs}:

% \subsubsection{Kiến trúc Hypertables}
% TimescaleDB giới thiệu khái niệm \textbf{Hypertables} - một lớp trừu tượng hóa (abstraction layer) giúp tự động phân mảnh dữ liệu (partitioning) theo thời gian và không gian. 
% \begin{itemize}
%     \item Đối với ứng dụng (NestJS), Hypertable hoạt động giống như một bảng SQL bình thường.
%     \item Tuy nhiên, ở tầng vật lý, dữ liệu được chia nhỏ thành các \textit{Chunks}. Khi thực hiện truy vấn dữ liệu lịch sử (ví dụ: vẽ biểu đồ nhiệt độ trong 7 ngày qua), TimescaleDB chỉ cần quét các Chunks liên quan thay vì quét toàn bộ bảng, giúp tăng tốc độ truy vấn lên hàng trăm lần so với PostgreSQL thuần.
% \end{itemize}

% \subsubsection{Khả năng nén và quản lý vòng đời dữ liệu}
% Trong IoT, dữ liệu cũ (ví dụ: dữ liệu từ năm ngoái) thường ít giá trị hơn dữ liệu mới. TimescaleDB cung cấp cơ chế:
% \begin{itemize}
%     \item \textbf{Data Retention Policy:} Tự động xóa dữ liệu quá cũ (ví dụ: sau 1 năm) để giải phóng dung lượng lưu trữ mà không cần chạy các cron job thủ công.
%     \item \textbf{Native Compression:} Nén dữ liệu lịch sử giúp giảm tới 90\% dung lượng đĩa cứng, tiết kiệm chi phí vận hành cho hệ thống.
% \end{itemize}

% \subsubsection{Tính thống nhất trong hệ sinh thái (SQL Compatibility)}
% Vì TimescaleDB bản chất là một Extension của PostgreSQL, nhóm nghiên cứu có thể sử dụng cùng một ngôn ngữ truy vấn \textbf{SQL} và cùng một driver kết nối cho cả dữ liệu nghiệp vụ (User/Farm - PostgreSQL) và dữ liệu cảm biến (Sensor Data - TimescaleDB). Điều này giúp giảm độ phức tạp khi phát triển và bảo trì hệ thống.